\documentclass["WS\space 16-17\space -\space LaTeX-Kurs\space -\space Praesentation\space -\space 4.tex"]{subfiles}

\begin{document}

\section{Fehler}
\begin{frame}[c]
	\begin{center}
		\LARGE \textbf{Fehler: Wie man damit fertig wird!}
	\end{center}
\end{frame}
%% -----------------------------------------------------------------------------------------------%
%% ------------------------------------------SUBSECTION-------------------------------------------%
%% -----------------------------------------------------------------------------------------------%
\subsection{Fehler im Code}
\begin{frame}[c]
	\begin{center}
		\large Fehler im Code
	\end{center}
\end{frame}
% -----------------------------------------------------------------------------------%
% ---------------------------------------MITTEILUNGEN--------------------------------%
% -----------------------------------------------------------------------------------%
\begin{frame}{Messages: {\LaTeX}'s Art zu reden}
    In der Regel zu finden im editoreigenen ``Message''-Fenster.\vspace{10pt}:
    \begin{columns}

      \column{0.5\textwidth}
      {\color{blue}\textbf{Warnings}}
      \begin{itemize}
      \item<2-> Oft blau hinterlegt.
        \note<2>{Zumindest weniger auffällig als rot.}
      \item<4-> Einzeln nicht kritisch für das Dokument.
      \item<7-> Häufungen führen zu unberechenbarem Verhalten.
      \item<7-> Kann oft ignoriert werden.
        \note<7>{Man darf sie aber durchaus auch fixen.}
        \note<7>{\\Je mehr Warnungen ignoriert werden, desto schwieriger ist es Ursachen von Fehlern zu finden (messages werden unpräzieser).}
      \end{itemize}

      \column{0.5\textwidth}
      {\color{red}\textbf{Errors}}
      \begin{itemize}
      \item<3-> Meist rot (auffällig) hinterlegt.
      \item<3-> Beginnen immer mit einem: {\color{red}\texttt{!}}
      \item<5-> Führen oft zum Abbruch der Kompilierung.
        \uncover<6->{\\\texttt{exited with status $\langle$zahl>0$\rangle$}}
      \item<8-> Vorsicht: Ignorieren garantiert späteren Frust.
        \note<8>{Ausnahmen bestätigen die Regel:}
        \note<8>{\\Durch \texttt{-interaction=nonstopmode} der meisten Editoren gibts meist doch output; ist das Dokument fertig, dann ist es halt fertig!}
      \end{itemize}
      
    \end{columns}

    \vspace{10pt}
    \uncover<9->{$\ldots$ und dann ist da noch die \texttt{.log}-Datei. Hier sammeln sich alle anderen Nachrichten.}
\end{frame}

\begin{frame}[fragile,c]{Falltest: Wir machen einen Fehler.}

  %\vspace{10pt}
  \begin{outputbox}
    Der folgende Code enthält einen Fehler:
  \end{outputbox}
  \vspace{8pt}
  
    \begin{lstlisting}
      \documentclass{scrartcl}

      \title{How to be a \LaTeX hacker}
      \author{Inge Inkompetent \& Anton Anfänger}

      \begin{document}

      \date{30. Februar 2017

        \maketitle

      \end{document}
    \end{lstlisting}
    
\end{frame}

\begin{frame}[fragile]{Aufbau einer Fehlermeldung}

  \begin{columns}

      \column{0.5\textwidth}
      % \textbf{Editorausgabe}
      \begin{block}{Editorausgabe}
      \note<1>{Editorausgabe hängt von Editor ab (die hier gezeigte ist mit texmaker generiert, dürfte aber eine durchschnittliche sein.)}
      \begin{lstlisting}[gobble=4]
        @@@Runaway argument?@
        @{11. Dezember 2013 @
          @! Paragraph ended before \date was complete.@
          @<to be read again> @
          @\par @
          @l.9@
        \end{lstlisting}
        \uncover<8>{
          \begin{block}{Weiterer Blick in die \texttt{.log}-Datei}
            \begin{tikzpicture}[baseline=(current bounding box.north)]
              \opaqueblock{8-}{visib}{\linewidth}{
                \texttt{I suspect you've forgotten a `\}', causing me to apply this
                  control sequence to too much text. How can we recover?
                  My plan is to forget the whole thing and hope for the best.}
              }
            \end{tikzpicture}
          \end{block}
        }
        %\uncover<8>{\textbf{Weiterer Blick in die \texttt{.log}-Datei}}
        
      \end{block}

      \column{0.5\textwidth}
      % \textbf{Erläuterung}
      \begin{block}{Erläuterung}
      \begin{itemize}
      \item<2-> \color<2>{blue!70!black}{{\LaTeX}'s eher kryptische Vermutung der Fehlerursache.}
        \note<2>{Diese Zeile kommt aus der \texttt{.log}-Datei}
      \item<3-> \color<3>{blue!70!black}{Zeile aus der \texttt{.tex}-Datei die den Fehler erzeugt.}
        \note<3>{Besser: Von der \LaTeX\ denkt, dass sie den Fehler erzeugt, manchmal ist der schon früher passiert.}
        \note<3>{\\Die Zeile ist ebenfalls aus der \texttt{.log}-Datei.}
      \item<4-> \color<4>{blue!70!black}{Eigentliche Fehlermeldung.}
        \note<4>{Eine neue Zeile hat nichts in einer Datumsangabe zu suchen, dass erkennt \LaTeX.}
      \item<5-> \color<5>{blue!70!black}{Mehr Kryptisches aus der \texttt{.log}-Datei.}
        \note<5>{Kp. ..vllt {\LaTeX}'s internes Vorgehen?}
      \item<6-> \color<6>{blue!70!black}{\TeX-Befehl in den \texttt{\newline} intern überführt wird (Bitte Vergessen!)}
      \item<7-> \color<7>{blue!70!black}{Zeilennummer im Quelltext (\texttt{.tex}-Datei) an der \LaTeX\ den Fehler vermutet.}
      \end{itemize}
      \end{block}
      
    \end{columns}
    
\end{frame}

\begin{frame}{Fehler im Code}
  \begin{block}{Häufige Fehler}
    \note<1->{Tabelle sollte nicht detailiert durchgesprochen werden. Sie sollte nur gesehen worden sein.}
    \colorlet{lstg}{green!70!black}

    \small
    \begin{tabular}[c]{l c r}
\kern-1em \cellcolor{blue!15}{\emph{Fehler}}    & \cellcolor{blue!15}{\emph{Fehlermeldung}}                                                 & \cellcolor{blue!15}{\emph{Lösung}}\\
\kern-1em Package fehlt                         & \color{red}{\texttt{! LaTeX Error: <}\textit{command}\texttt{> undefined.}}               & \texttt{$\backslash$\color{lstg}{usepackage}%
                                                                                                                                              \color{black}{\{<}}\textit{package}\texttt{>\}}\\
\kern-1em \cellcolor{blue!5}Verschrieben (cmd)  & \cellcolor{blue!5}\color{red}{\texttt{! LaTeX Error: <}\textit{command}\texttt{> undefined.}} & \cellcolor{blue!5}Korrektur\\
\kern-1em Verschrieben (file)                   & \color{red}{\texttt{! LaTeX Error: File <}\textit{path}\texttt{> not found}}              & Dateipfad prüfen.\\
\kern-1em \cellcolor{blue!5}Kein Mathe-Kontext  & \cellcolor{blue!5}\color{red}{\texttt{! Missing \$ inserted.}}                            & \cellcolor{blue!5}Ergänze \color{lstg}{\texttt{\$}}\color{black}%
                                                                                                                                              \ oder \color{lstg}\texttt{align}\color{black}-env.\\
\kern-1em \color{lstg} \} \color{black} fehlt od. nach Umbruch & \color{red}{\texttt{! Par. ended before macro was compl.}}                 & Makro vor "\texttt{$\backslash$\color{lstg}{end}}\color{black}"%
                                                                                                                                              \ schließen.\\
\kern-1em \cellcolor{blue!5}Env nicht geschl.   & \cellcolor{blue!5}\color{red}{\texttt{! Emrgncy stop err\_env-n-closed.tex}}              & \cellcolor{blue!5}\texttt{$\backslash$\color{lstg}{end}%
                                                                                                                                              \color{black}{\{<}}\textit{env}\texttt{>\}} vor %
                                                                                                                                              "\texttt{$\backslash$\color{lstg}{end}}\color{black}".
    \end{tabular}

    \begin{block}{Zusatz}
      Dies Tabelle ist hoffentlich nützlich um für die häufigsten Fehler zu sensibilisieren oder Fehlermeldungen nachzuschlagen.
      Sie zu lernen hat keinen Sinn. Einiges in der Tabelle ist abgekürzt, es ist aber hoffentlich zu erraten was jeweils gemeint ist.
      Wenn von \textit{Makro} oder \textit{Environment} (=Umgebungen) die Rede ist, dann sind nicht etwa alle Verteter dieser Gattung gemeint,
      jeder Befehl verhält sich anders und erzeugt den Fehler in anderen Situationen.\\
      "\texttt{$\backslash$\color{lstg}{end}}" meint eine Instanz an der environments schließen; Häufig \texttt{$\backslash$\color{lstg}{end}\color{black}{\{document\}}} aber auch übrig gebliebene
      \texttt{$\backslash$\color{lstg}{end}}\color{black} von gelöschten Umgebungen.
    \end{block}
    
  \end{block}
\end{frame}
%% -----------------------------------------------------------------------------------------------%
%% ------------------------------------------EXTERNAL ERROR---------------------------------------%
%% -----------------------------------------------------------------------------------------------%
\subsection{Externe Fehlerquellen}
\begin{frame}[c]
	\begin{center}
		\large Externe Fehlerquellen
	\end{center}
\end{frame}
% -----------------------------------------------------------------------------------%
% ---------------------------------------FRAME---------------------------------------%
% -----------------------------------------------------------------------------------%
\begin{frame}{Fehler außerhalb des Codes:}
  \begin{outputbox}
    Diese Fehler sind schwierig ausfindig zu machen, da es \textbf{keine präzise Fehlermeldung} gibt.\\
    Entweder beschwert sich \LaTeX über eine Stelle im Code die nicht direkt mit dem eigentlichen Fehler zu tun hat,\\
    oder die Fehlermeldung bezieht sich auf eine der externen Dateien (mit unbekannter Syntax) und ist daher meist unverständlich.
  \end{outputbox}

  \begin{columns}

    \column{0.7\textwidth}
    \begin{block}{Problem}
      \begin{itemize}
      \item<2-> Encoding-Probleme
        \begin{itemize}
        \item<3-> Verschiedene OS's (Partner oder Wechsel)
        \item<3-> Minderwertiges Encoding gewählt
        \end{itemize}
      \item<4-> Veraltete Packete
        \begin{itemize}
        \item<5-> {\TeX}live / Mik\TeX ( / Mac\TeX ) nicht up-to-date
        \item<7-> Nutzung überholter Befehle
        \end{itemize}
      \item<9-> Bibliography
      \end{itemize}
    \end{block}
    
    \column{0.3\textwidth}
    \begin{block}{Lösung}
      \begin{itemize}
      \item<4-> UTF-8 ist meißt ein guter Rat.
      \item<6-> Update der \LaTeX-Distribution
      \item<8-> Experimentieren mit \texttt{nag.sty}
      \item<10-> Syntax-Check in \texttt{.bib}-Datei. 
      \end{itemize}
      
    \end{block}
    
  \end{columns}
  
\end{frame}
%% -----------------------------------------------------------------------------------------------%
%% ------------------------------------------SECTION-------------------------------------------%
%% -----------------------------------------------------------------------------------------------%
\subsection{Lösungsstrategien}
\begin{frame}[c]
	\begin{center}
		\large Lösungsstrategien
	\end{center}
\end{frame}
% -----------------------------------------------------------------------------------%
% ---------------------------------------FRAME---------------------------------------%
% -----------------------------------------------------------------------------------%
\begin{frame}{Allgemeines}

    \begin{block}{Rule34 of \LaTeX}
      \uncover<2->{Dir schwebt eine nützliche Funktion vor?}
      \begin{itemize}
      \item<3-> Es gibt bestimmt ein Package dafür. 
      \end{itemize}
    \end{block}

    \uncover<4->{
      \begin{block}{Einrückung}
        Übersicht im Quellcode beugt Fehlern vor.
        \begin{itemize}
        \item<5-> Tabellen (Bsp. mit \%$\backslash$newline einfügen)
        \item<6-> Leerzeichen an der falschen Stelle Fehlermeldungen!
        \end{itemize}
      \end{block}
    }

\end{frame}

\begin{frame}[c]{Debug-Strategie: Seven C}
  \begin{block}{Fehlersuche:}

    \begin{outputbox}
      Bewährtes Vorgehen, Wenn \LaTeX\ sich beschwert und die Meldung unverständlich ist:
    \end{outputbox}
    
    \begin{enumerate}
      \addtocounter{enumi}{-1}
    \item \textbf{Compile} often.\\
      \only<2>{(Anfangs) Fehlersuche abkürzen durch häufiges Kompilieren.}
    \item \textbf{Check} for common errors.\\
      \only<3>{Stehen Befehle im falschen Kontext (mathe-befehle im Text o. a. environments).}
    \item \textbf{Control} your spelling.\\
      \only<4>{Kontrolle der Schreibweise benutzter Makros, Dateien, etc.}
      \note<4>{Hinweis: Viele Editoren bieten hierfür Kürzel zum Blockweisen auskommentieren.}
    \item \textbf{Comment} recent stuff.\\
      \only<5>{Letztverfasstes absatzweise auskommentieren bis zu erfolgreicher Kompilierung.}
      \note<5>{Hinweis: Viele Editoren bieten hierfür Kürzel zum Blockweisen auskommentieren.}
    \item \textbf{Cearch} via search-engines.\\
      \only<6>{Google your error: Viele Probleme wurden schon in Foren wie \texttt{tex.stackexchange.com} behandelt.}
      \note<6>{Blick in die .log-datei?!}
      \note<6>{\\Datum bei Foreneinträgen beachten! (Viele überholte packages in Verkehr)}
      \note<6>{\\Besonders bei copy-\&paste-Verfahren!}
    \item \textbf{CTAN} consultation.\\
      \only<7>{Wenn der Fehler eingegrenzt aber nicht verstanden ist:\\
        Dokumentationen im \texttt{.pdf}-Format gibt es für alle packages auf dem \textit{Comprehensive \TeX\ Archive Network} (CTAN)\\
        Oft sind diese in Unterordnern der \LaTeX- (eig. Miktex- bzw texlive-) Installation bereits vorhanden.\\
        \textit{screenshots einfügen}}
      \note<7>{Auf texdoc-Funktionalität hinweisen (Komandozeile \texttt{texdoc})}
    \item \textbf{Call} for help.
    \end{enumerate}
  \end{block}
\end{frame}

\begin{frame}[fragile]{Ideen zur Partnerarbeit}

  \begin{enumerate}
    \item<1-> Verständigung über Aufteilung der Themen
      \note<1>{Grundlagen / Auswertung ..oder thematisch... mal so mal so .. bla bla.}
    \item<2-> Erstellen der groben Dokumentstruktur mit \lstinline|\input{}|
    \item<4-> Dropbox (oder besseres Versions-Kontroll-System [VCS] )
      \note<4> {Subversion, mercurial, bla ----> GIT}
  \end{enumerate}

  %\setbeamerfont{alerted txt}{series=\bfseries,family=\ttfamily}
  %\setbeamercolor{lstlisting}{fg=black,bg=blue!10}
  \uncover<2->{\Code}
  % \begin{beamercolorbox}{lstlisting}
  %   \begin{semiverbatim} \small
  %     \uncover<2->{\color{gray}%Strukturskellet des Berichts\color{black}}
  %     \uncover<2->{\\\color{green!70!black}input\color{black}\{gemeinsamer_header.tex\}}

  %     \uncover<2->{\\\color{green!70!black}begin\color{black}\{document\}}
        
  %     \uncover<2->{\\\color{green!70!black}input\color{black}\{document\}}
        
  %   \end{semiverbatim}
  % \end{beamercolorbox}
  \begin{fancycolorbox}
    \opaqueblock{3-}{visib}{\textwidth}{
    \begin{semiverbatim} \small
      \color{gray}\%Strukturskellet des Berichts
      \newline\vspace{9pt}\color{black}
      \\\color{green!70!black}input\color{black}\{gemeinsamer\_header.tex\}
      \newline\vspace{7pt}
      \\\color{green!70!black}begin\color{black}\{document\}
      \newline\vspace{7pt}
      \\\color{green!70!black}input\color{black}\{Partner1.tex\}
      \newline\vspace{7pt}
      \\\color{green!70!black}input\color{black}\{Partner2.tex\}
      \newline\vspace{7pt}
      \\\color{green!70!black}end\color{black}\{document\}
    \end{semiverbatim}
    }
  \end{fancycolorbox}
 \note<3>{header.tex ins selbe Verzeichnis (oder alternativen pfad angeben).}

\end{frame}

\end{document}