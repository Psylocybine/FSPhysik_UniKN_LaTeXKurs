%	Textshapes (Fett, Kursiv, Slanted, Kapitälchen,..)
%
%	Größen (tiny, footnotesize, small, normalsize, large Large LARGE, huge)
%
%	Newline ()
%
%	Normaler absatz (Editorspezifisch?)
%
%	Pagebreak (newpage)
%
%	vspace{Abstand}
%
%	Zum Schluss: reservierte Zeichen (+Kommentare)
%
%	centering
%
%	Überschriften, Sections, Subsections (optionales Argument)
%
%	Inhaltsverzeichnis
\section{Textsatz}
\begin{frame}[c]
	\begin{center}
		\huge \textbf{Grundlegendes zum Textsatz}
	\end{center}
\end{frame}
%-----------------------------------------------------------------------------------------------%
%------------------------------------------SUBSECTION-------------------------------------------%
%-----------------------------------------------------------------------------------------------%
\subsection{Textmodifikatoren}
\begin{frame}[c]
	\begin{center}
		\large Textmodifikatoren
	\end{center}
\end{frame}
%-----------------------------------------------------------------------------------%
%---------------------------------------FRAME---------------------------------------%
%-----------------------------------------------------------------------------------%
\begin{frame}[fragile]
	Dokument:
	\begin{lstlisting}[breaklines=true,gobble=4]
        \documentclass[xcolor=table,dvipsnames,8pt,t]{beamer}

\newcommand\hmmax{0}		%	damit ein son dummer Fehler nicht kommt :D
\newcommand\bmmax{0}

%--------------------------!-------------------------------------%
%--------------------------Notizen im PDF------------------------%
%----------------------------------------------------------------%

\setbeamertemplate{note page}[plain]										% Notizenseiten nur mit gesetztem Text
\AtBeginNote{\let\enumerate\itemize\let\endenumerate\enditemize\LARGE\par}	% Itemize in Notizen mit "Bullet", und \LARGE
\AtEndNote{\par}

%\setbeameroption{hide notes} % Only slides
%\setbeameroption{show only notes} % Only notes

%\usepackage{pgfpages}
%\setbeameroption{show notes on second screen=bottom} % Both

\makeatletter																% Zu jeder Slide eine Notizenseite
\def\beamer@framenotesbegin{% at beginning of slide
	\gdef\beamer@noteitems{}%
	\gdef\beamer@notes{{}}% used to be totally empty.
}
\makeatother

%--------------------------!-------------------------------------%
%--------------------------Schriftart----------------------------%
%----------------------------------------------------------------%

%\usepackage{mathptmx}
\usepackage{helvet}

%--------------------------!-------------------------------------%
%--------------------------Beamer Einstellungen------------------%
%----------------------------------------------------------------%

\usetheme{Dresden}
\usecolortheme{beaver}

\setbeamertemplate{section in toc}	[sections numbered]
\setbeamertemplate{theorems}		[numbered]

\setbeamertemplate{itemize items}	[-]

\setbeamertemplate{caption}			[numbered]

%--------------------------!-------------------------------------%
%--------------------------Encoding------------------------------%
%----------------------------------------------------------------%

\usepackage[ngerman]{babel}			%	Einstellen der Sprache
\usepackage[T1]		{fontenc}		%	Wie wird Text ausgegeben, d.h. im PDF
\usepackage[utf8]	{inputenc}		%	Welche Zeichen 'versteht' LaTeX bei der Eingabe?
									%	Die beiden Pakete fontenc und inputenc gehören zwar nicht zusammen aber es wird empfohlen erst fontenc und dann inputenc zu laden
									%	http://tex.stackexchange.com/questions/44694/fontenc-vs-inputenc

%--------------------------!-------------------------------------%
%--------------------------Mathe---------------------------------%
%----------------------------------------------------------------%

\usepackage[fleqn]{mathtools}		%	Erweiterung von AMSMath, lädt automatisch AMSMath - für viele Mathe-Werkzeuge, fleqn ist für Mathe linksbündig
\usepackage{amssymb}				%	Für eine Vielzahl an mathematischen Symbolen

%--------------------------!-------------------------------------%
%--------------------------Listings------------------------------%
%----------------------------------------------------------------%

\usepackage{listings}
	\lstset{
		language		= [LaTeX]TeX,									%	[dialect]language
		extendedchars	= true,
		inputencoding	= utf8,
		upquote			= false,										%	benötigt textcomp bei true
		%
		backgroundcolor	= \color{blue!10},
		%
		basicstyle		= \ttfamily\footnotesize,
		identifierstyle	= ,
		commentstyle	= \color{gray},
		stringstyle		= \color{black},
		keywordstyle	= \color{green!70!black},						%	[] optional argument for level it applies to
		%directivestyle	= \normalfont\color{green},
		%
		showstringspaces= false,
		showspaces		= false,
		showtabs		= false,
		tab				= \rightarrowfill,
		tabsize			= 4,
		%
		linewidth			= \linewidth,
		breaklines			= true,
		breakautoindent		= true,
		breakindent			= 0pt,
		postbreak			= ,
		breakatwhitespace	= true,
		%
		%makemacrouse	= false,
		%
		aboveskip		= 0.2cm,
		belowskip		= 0.2cm,
		xleftmargin		= 0.1cm,
		xrightmargin	= 0.1cm,
		%
		frame			= single,									%	 none, leftline, topline, bottomline, lines (top and bottom), single, shadowbox ODER t,r,b,l
		%frameround		= ,
		framesep		= 0.20cm,
		rulesep			= 0.02cm,
		framerule		= 0.0cm,
%		framexleftmargin	= ,
%		framexrightmargin	= ,
%		framextopmargin		= ,
%		framexbottommargin	= ,
%		rulecolor		= ,
%		fillcolor		= ,
%		rulesepcolor	= ,	
		%
		columns			= fixed,
		fontadjust		= false,
		%
		%morestring		= [d][]$,										%	d - double, b - backslash
		morecomment		= [s][\color{orange}]{$}{$},					%	l - line, s - open and close,n - nested open and close, i - invisible
%		
		literate		= {Ö}{{\"O}}1
						  {Ä}{{\"A}}1
						  {Ü}{{\"U}}1
						  {ß}{{\ss}}1
						  {ü}{{\"u}}1
						  {ä}{{\"a}}1
						  {ö}{{\"o}}1
						  {\&}{{{\color{blue}{\&}}}}{1},
%
		morekeywords	= {part,chapter,section,subsection,subsubsection,paragraph},
		morekeywords	= {tableofcontents,listoffigures,listoftables},
		morekeywords	= {includegraphics,textSC,text,mathbb,eqref},,
		morekeywords	= {SI,SIrange,SIlist,num,per,tothe,metre,second,kilogram},
		morekeywords	= {printbibliography,citeauthor,citeyear,citetitle,psq,psqq},
		morekeywords	= {align},
		moredelim		= **[is][\only<+>{\color{red}}]{@}{@}
	}

%--------------------------!-------------------------------------%
%--------------------------SIUNITX-------------------------------%
%----------------------------------------------------------------%

\usepackage{siunitx}
	\sisetup{detect-all					= false}		%	Passt die Schriftart der gesetzen Einheiten / Zahlen an
	
	%\sisetup{detect-display-math		= true}
	%\sisetup{detect-family				= true}
	%\sisetup{detect-shape				= true}
	%\sisetup{detect-weight				= true}
	
	\sisetup{group-separator			=	\,}			%	Zeichen Zwischen Dreiergruppen in langen Zahlen (1.000.000,5)
	
	\sisetup{separate-uncertainty		=	false}		%	Unsicherheiten mit +- (true) oder mit anderen Zeichen (false)
	\sisetup{uncertainty-separator		=	\,}			%	Zeichen zwischen Zahl und Unsicherheit bei seperate uncertainty=false
	\sisetup{output-open-uncertainty	=	(}			%	Zeichen vor Unsicherheit bei seperate uncertainty=false
	\sisetup{output-close-uncertainty	=	)}			%	Zeichen nach Unsicherheit bei seperate uncertainty=false
	
	\sisetup{list-final-separator	=	{ und }}		%	Finales Trennzeichen bei Listen
	\sisetup{list-pair-separator	=	{ und }}		%	Trennzeichen bei Listen bestehend aus zwei Teilen
	\sisetup{list-separator			=	{, }}			%	Trennzeichen bei Listen aus 3 oder mehr Teilen
	\sisetup{range-phrase			=	{ bis }}		%	Text zur angabe von Bereichen (500nm bis 700nm)
	
	\sisetup{inter-unit-product	=	\ensuremath{\,}}	%	Zeichen zwischen Einheitenzeichen
	
	\sisetup{per-mode					=	reciprocal-positive-first}	%	Ausgabe von Einheiten (symbol-or-fraction, fraction, symbol, reciprocal-positive-first, repeated-symbol)
	\sisetup{per-symbol					=	/}			%	Zeichen zwischen Zähler und Nenner bei Option \symbol
	\sisetup{bracket-unit-denominator	=	false}		%	Klammern um Zähler und Nenner oder nicht
	
	\sisetup{number-unit-product=	\ensuremath{\,}}	%	Symbol zwischen Zahl und Einheit
	
	\sisetup{multi-part-units			=	brackets}	%	Angabe der Einheit bei Zahlen mit Unsicherheit (z.B. mit '+') (brackets, repeat, single)
	\sisetup{product-units				=	repeat}		%	Angabe der Einheit bei Produkten (Eingabe mit 'x' oder '/') (brackets, repeat, single)
	\sisetup{list-units					=	repeat}		%	Angabe der Einheit bei Listen
	\sisetup{range-units				=	repeat}		%	Angabe der Einheit bei Bereichen 
		
	\sisetup{exponent-product			=	\cdot}		%	Zeichen bei 1*10^5 zwischen Zahl und 10^5
	
	%\sisetup{locale = DE}								%	Automatische Einstellung der Ausgabe für bestimmte Regionen (UK, US, DE, FR, ZA)
	%\sisetup{strict = true}							%	Überschreibt alle manuell gesetzten Einstellungen und benutzt nur die durch 'locale' gesetzen Werte

%--------------------------!-------------------------------------%
%--------------------------Abstände------------------------------%
%----------------------------------------------------------------%

\setlength{\parindent}				{0cm}							%	Längenangabe für die Einrückung der ersten Zeile eines neuen Absatzes.
\setlength{\parskip}				{2mm plus 1mm minus 1mm}		%	Längenangabe für den Abstand zwischen zwei Absätzen.

%--------------------------!-------------------------------------%
%--------------------------Color---------------------------------%
%----------------------------------------------------------------%

\usepackage{color}

%--------------------------!-------------------------------------%
%--------------------------TikZ----------------------------------%
%----------------------------------------------------------------%

\usepackage{tikz} % needed for opaqueblock-definition needed for hiding colorboxes

% \usepackage[table]{xcolor} %needed for command cellcolor %moved into beamer-option

\tikzset{visib/.style={rectangle,color=blue,fill=blue!10,text=black,text opacity=1, text width=#1,align=flush center}}
\tikzset{invisib/.style={rectangle,color=white,fill=white,text=black,text opacity=0, text width=#1,align=flush center}}

\newenvironment{fancycolorbox}{\begin{center}
\begin{tikzpicture}}{\end{tikzpicture}
\end{center}}

\newcommand{\opaqueblock}[4]{
\node<#1>[#2=#3] (X) {#4};
}

%--------------------------!-------------------------------------%
%--------------------------Chemisches Zeug-----------------------%
%----------------------------------------------------------------%

\usepackage{chemmacros}
\usechemmodule{all}

%--------------------------!-------------------------------------%
%--------------------------Akronyme------------------------------%
%----------------------------------------------------------------%

\usepackage{acro} 

\DeclareAcronym{D}{
      short = D ,
      long = Wüste ,
      long-plural = n ,
      foreign = Desert ,
      short-format = \scshape 
      } 

%--------------------------!-------------------------------------%
%--------------------------Picture Stuff-------------------------%
%----------------------------------------------------------------%

%for watermarks (distinguish between math and text)
  
%\usepackage[left=1in,right=1in]{geometry}
%\usepackage{xcolor,rotating,picture}

%--------------------------!-------------------------------------%
%--------------------------Tabellen------------------------------%
%----------------------------------------------------------------%

\usepackage{booktabs}

%--------------------------!-------------------------------------%
%--------------------------Literatur-----------------------------%
%----------------------------------------------------------------%

\usepackage{csquotes}									%	Wird von BibLaTeX benötigt
\usepackage[backend=biber,style=numeric]{biblatex}		%	Für die Literaturverwaltung
	\addbibresource{Literaturverzeichnis.bib}			%	Laden eures Literaturverzeichnisses

% enable compiling tex-files that are "inputed" via \subfile-macro and subfiles-documentclass
\usepackage[]{subfiles}
  
\setbeamertemplate{bibliography item}[text]

%--------------------------!-------------------------------------%
%--------------------------Newcommands---------------------------%
%----------------------------------------------------------------%

\usepackage{suffix}

%\renewcommand{\epsilon}{\varepsilon}
\newcommand{\textSC}[1]{{\normalfont  \textsc{#1}}} 
\renewcommand{\quote}[1]{\glq{#1}\grq}        			%	Um Worte EINFACH in Anführungszeichen zu setzen
\newcommand{\qquote}[1]{\glqq{#1}\grqq}  				%	Um Worte EINFACH in doppelte Anführungszeichen zu setzen

\newcounter{AufgCounter}
\setcounter{AufgCounter}{1}
\numberwithin{AufgCounter}{section}
\resetcounteronoverlays{AufgCounter}

\definecolor{AufgColor}{HTML}{52B0E3}

\newcommand{\Aufgabee}	{\refstepcounter{AufgCounter}	\textbf{{\large 	 \color{AufgColor} Aufgabe 		\theAufgCounter:}}\vspace{0.1cm}\newline}
\newcommand{\Losung}	{								\textbf{{\normalsize \color{AufgColor} L{\"o}sung 	\theAufgCounter:}}\vspace{-0.3cm}\newline}
\newcommand{\Code}		{								\textbf{{\normalsize \color{AufgColor} Code 		\theAufgCounter:}}\vspace{-0.4cm}\newline}

\WithSuffix\newcommand\Aufgabee*	{\textbf{{\large 	  \color{AufgColor} Aufgabe:}}		\vspace{0.1cm}\newline}
\WithSuffix\newcommand\Losung*		{\textbf{{\normalsize \color{AufgColor} L{\"o}sung:}}	\vspace{-0.3cm}\newline}
\WithSuffix\newcommand\Code*		{\textbf{{\normalsize \color{AufgColor} Code:}}			\vspace{-0.4cm}\newline}

\newcommand{\Ausgabe}	            {\textbf{{\normalsize \color{AufgColor} Ausgabe:}}      \vspace{0.1cm}\newline}
\newcommand{\Befehle}	            {\textbf{{\normalsize \color{AufgColor} Befehle:}}      \vspace{-0.3cm}}

\newcommand{\btVFill}{\vskip0pt plus 1filll}			%	Um Sachen ans Ende der Folie zu schreiben

\newcommand{\linebreakrule}{\vspace{-0.35cm}\makebox[\textwidth]{\rule{\textwidth}{0.2mm}}}

%\newcolumntype{K}[1]{>{\centering\arraybackslash}p{#1}}

%--------------------------!-------------------------------------%
%--------------------------Newenvironment------------------------%
%----------------------------------------------------------------%

\newtheorem{thmex}	{Aufgabe}	[section]

\newenvironment{Aufgabe}
	{\begin{thmex}\vspace{-0.1cm}}
	{\end{thmex}}

\setbeamercolor{block body example}	{bg=yellow!10}
\setbeamercolor{postit}				{fg=black,bg=yellow!10}

\newenvironment{outputbox}
	{\vspace{0.3cm}\ \\\begin{beamercolorbox}[sep=0cm,colsep=0.1cm,colsep*=0.1cm,wd=\textwidth]{postit}\vspace{-0.3cm}\\\normalfont}
	{\end{beamercolorbox}}
        
        \begin{document}
        
        
        \end{document}
	\end{lstlisting}
	\pause\btVFill
	\Aufgabee
	Schreibe zwischen \lstinline[basicstyle=\normalfont\ttfamily\normalsize]|\begin{document}| und \lstinline[basicstyle=\normalfont\ttfamily\normalsize]|\end{document}| folgenden Text:
	\begin{outputbox}
		Bei der Heisenberg-Methode lassen sich Ort und Geschwindigkeit eines bewegten Löwen nicht gleichzeitig bestimmen. Da bewegte Löwen in der Wüste keinen physikalisch sinnvollen Ort einnehmen, eignen sie sich auch nicht zur Jagd. Die Löwenjagd kann sich demnach zu 100 Prozent auf ruhende Löwen beschränken. Das Fangen eines ruhenden, bewegungslosen Löwen wird dem Leser als Übungsaufgabe überlassen.
	\end{outputbox}
	\vspace{0.2cm}
\end{frame}
%-----------------------------------------------------------------------------------%
%---------------------------------------FRAME---------------------------------------%
%-----------------------------------------------------------------------------------%
\begin{frame}[fragile]
	\Befehle\vspace{0.1cm}
	\begin{center}
		\begin{tabular}{lll}
			\toprule
			\LaTeX\ Befehl						&	Aussehen				&	Modifikation	\\ \midrule
			\lstinline|\textit{Beispieltext}|	&	\textit{Beispieltext}	&	Kurisv			\\
			\lstinline|\textbf{Beispieltext}|	&	\textbf{Beispieltext}	&	Fett			\\
			\lstinline|\textsc{Beispieltext}|	&	\textsc{Beispieltext}	&	Kapitälchen 		\\
			\lstinline|\textSC{Beispieltext}|	&	\textSC{Beispieltext}	&	Kapitälchen in Überschriften		\\ 
			&		&	{\tiny (nicht im Allgemeinen, siehe Kapitel \qquote{Newcommand})}		\\   \bottomrule
		\end{tabular}
	\end{center}
	\btVFill
	\Aufgabee
	Setze \qquote{Löwe} \textbf{fett}, \qquote{Ort} und \qquote{Geschwindigkeit} \textit{kursiv} sowie \qquote{\textsc{Heisenberg}} in Kapitälchen:
	\begin{outputbox}
		Bei der \textsc{Heisenberg}-Methode lassen sich \textit{Ort} und \textit{Geschwindigkeit} eines bewegten \textbf{Löwen} nicht gleichzeitig bestimmen. Da bewegte \textbf{Löwen} in der Wüste keinen physikalisch sinnvollen \textit{Ort} einnehmen, eignen sie sich auch nicht zur Jagd. Die \textbf{Löwenjagd} kann sich demnach zu 100 Prozent auf ruhende \textbf{Löwen} beschränken. Das Fangen eines ruhenden, bewegungslosen \textbf{Löwen} wird dem Leser als Übungsaufgabe überlassen.
	\end{outputbox}
	\vspace{0.3cm}
\end{frame}
%-----------------------------------------------------------------------------------%
%---------------------------------------FRAME---------------------------------------%
%-----------------------------------------------------------------------------------%
\begin{frame}[fragile]
	\Losung
	\begin{outputbox}
			Bei der \textsc{Heisenberg}-Methode lassen sich \textit{Ort} und \textit{Geschwindigkeit} eines bewegten \textbf{Löwen} nicht gleichzeitig bestimmen. Da bewegte \textbf{Löwen} in der Wüste keinen physikalisch sinnvollen \textit{Ort} einnehmen, eignen sie sich auch nicht zur Jagd. Die \textbf{Löwenjagd} kann sich demnach zu 100 Prozent auf ruhende \textbf{Löwen} beschränken. Das Fangen eines ruhenden, bewegungslosen \textbf{Löwen} wird dem Leser als Übungsaufgabe überlassen.
	\end{outputbox}

	\Code
	\begin{lstlisting}
Bei der \textsc{Heisenberg}-Methode lassen sich \textit{Ort} und \textit{Geschwindigkeit} eines bewegten \textbf{Löwen} nicht gleichzeitig bestimmen. Da bewegte \textbf{Löwen} in der Wüste keinen physikalisch sinnvollen \textit{Ort} einnehmen, eignen sie sich auch nicht zur Jagd. Die \textbf{Löwenjagd} kann sich demnach zu 100 Prozent auf ruhende \textbf{Löwen} beschränken. Das Fangen eines ruhenden, bewegungslosen \textbf{Löwen} wird dem Leser als Übungsaufgabe überlassen.
	\end{lstlisting}
\end{frame}
%-------------------------------------------------------------------------------------------------%
%------------------------------------------SUBSECTION---------------------------------------------%
%-------------------------------------------------------------------------------------------------%
\subsection{Strukturierung innerhalb eines Abschnitts}
\begin{frame}[c]
	\begin{center}
		\large Strukturierung innerhalb eines Abschnitts
	\end{center}
\end{frame}
%-----------------------------------------------------------------------------------%
%---------------------------------------FRAME---------------------------------------%
%-----------------------------------------------------------------------------------%
\begin{frame}[fragile]
	\Befehle
	\begin{center}
		\begin{tabular}{lll}
			\toprule
			\LaTeX\ Befehl								&	Funktion			\\ \midrule
			\lstinline|\newline| oder \lstinline|\\|	&	Zeilenumbruch		\\
			\lstinline|\newpage|						&	Seitenumbruch		\\
			eine leere Zeile							&	Neuer Absatz		\\ \bottomrule
		\end{tabular}
	\end{center}
	\pause\btVFill
	\Aufgabee
	Füge nach \qquote{Jagd} einen neuen Absatz ein: 
	\begin{outputbox}
		Bei der \textsc{Heisenberg}-Methode lassen sich \textit{Ort} und \textit{Geschwindigkeit} eines bewegten \textbf{Löwen} nicht gleichzeitig bestimmen. Da bewegte \textbf{Löwen} in der Wüste keinen physikalisch sinnvollen \textit{Ort} einnehmen, eignen sie sich auch nicht zur Jagd.
		
		Die \textbf{Löwenjagd} kann sich demnach zu 100 Prozent auf ruhende \textbf{Löwen} beschränken. Das Fangen eines ruhenden, bewegungslosen \textbf{Löwen} wird dem Leser als Übungsaufgabe überlassen.
	\end{outputbox}
	\vspace{0.3cm}
\note[item]<1>{statt leerer Zeile kann auch "par" verwendet werden}
\end{frame}
%-----------------------------------------------------------------------------------%
%---------------------------------------FRAME---------------------------------------%
%-----------------------------------------------------------------------------------%
\begin{frame}[fragile]
	\Losung
	\begin{outputbox}
		Bei der \textsc{Heisenberg}-Methode lassen sich \textit{Ort} und \textit{Geschwindigkeit} eines bewegten \textbf{Löwen} nicht gleichzeitig bestimmen. Da bewegte \textbf{Löwen} in der Wüste keinen physikalisch sinnvollen \textit{Ort} einnehmen, eignen sie sich auch nicht zur Jagd.
		
		Die \textbf{Löwenjagd} kann sich demnach zu 100 Prozent auf ruhende \textbf{Löwen} beschränken. Das Fangen eines ruhenden, bewegungslosen \textbf{Löwen} wird dem Leser als Übungsaufgabe überlassen.
	\end{outputbox}

	\Code
	\begin{lstlisting}[breaklines=true]
Bei der \textsc{Heisenberg}-Methode lassen sich \textit{Ort} und \textit{Geschwindigkeit} eines bewegten \textbf{Löwen} nicht gleichzeitig bestimmen. Da bewegte \textbf{Löwen} in der Wüste keinen
physikalisch sinnvollen \textit{Ort} einnehmen, eignen sie sich auch nicht zur Jagd. \\ \\
Die \textbf{Löwenjagd} kann sich demnach zu 100 Prozent auf ruhende \textbf{Löwen} beschränken. Das Fangen eines ruhenden, bewegungslosen \textbf{Löwen} wird dem Leser als Übungsaufgabe überlassen.
	\end{lstlisting}
\end{frame}
%-----------------------------------------------------------------------------------%
%---------------------------------------FRAME---------------------------------------%
%-----------------------------------------------------------------------------------%
\begin{frame}[fragile]
	\Aufgabee
	Schiebe den zweiten Absatz auf eine neue Seite:
	\begin{outputbox}
		Bei der \textsc{Heisenberg}-Methode lassen sich \textit{Ort} und \textit{Geschwindigkeit} eines bewegten \textbf{Löwen} nicht gleichzeitig bestimmen. Da bewegte \textbf{Löwen} in der Wüste keinen physikalisch sinnvollen \textit{Ort} einnehmen, eignen sie sich auch nicht zur Jagd. 
	\end{outputbox}
	\linebreakrule
	\begin{outputbox}
		Die \textbf{Löwenjagd} kann sich demnach zu 100 Prozent auf ruhende \textbf{Löwen} beschränken. Das Fangen eines ruhenden, bewegungslosen \textbf{Löwen} wird dem Leser als Übungsaufgabe überlassen.
	\end{outputbox}

	\btVFill\Befehle
	\begin{center}
		\begin{tabular}{lll}
			\toprule
			\LaTeX\ Befehl								&	Funktion			\\ \midrule
			\lstinline|\newline| oder \lstinline|\\|	&	Zeilenumbruch		\\
			\lstinline|\newpage|						&	Seitenumbruch		\\
			eine leere Zeile		&	Neuer Absatz		\\ \bottomrule
		\end{tabular}
	\end{center}
	\vspace{0.1cm}
\note[item]{Querstrich steht für neue Seite}
\end{frame}
%-----------------------------------------------------------------------------------%
%---------------------------------------FRAME---------------------------------------%
%-----------------------------------------------------------------------------------%
\begin{frame}[fragile]
	\Losung
	\begin{outputbox}
		Bei der \textsc{Heisenberg}-Methode lassen sich \textit{Ort} und \textit{Geschwindigkeit} eines bewegten \textbf{Löwen} nicht gleichzeitig bestimmen. Da bewegte \textbf{Löwen} in der Wüste keinen physikalisch sinnvollen \textit{Ort} einnehmen, eignen sie sich auch nicht zur Jagd. 
	\end{outputbox}
    \linebreakrule
	\begin{outputbox}
		Die \textbf{Löwenjagd} kann sich demnach zu 100 Prozent auf ruhende \textbf{Löwen} beschränken. Das Fangen eines ruhenden, bewegungslosen \textbf{Löwen} wird dem Leser als Übungsaufgabe überlassen.
	\end{outputbox}

	\Code\btVFill
	\begin{lstlisting}
Bei der \textsc{Heisenberg}-Methode lassen sich \textit{Ort} und \textit{Geschwindigkeit} eines bewegten \textbf{Löwen} nicht gleichzeitig bestimmen. Da bewegte \textbf{Löwen} in der Wüste keinen physikalisch sinnvollen \textit{Ort} einnehmen, eignen sie sich auch nicht zur Jagd. 

\newpage
Die \textbf{Löwenjagd} kann sich demnach zu 100 Prozent auf ruhende \textbf{Löwen} beschränken. Das Fangen eines ruhenden, bewegungslosen \textbf{Löwen} wird dem Leser als Übungsaufgabe überlassen.
	\end{lstlisting}
	\vspace{0.3cm}
\end{frame}
%-------------------------------------------------------------------------------------------------%
%------------------------------------------SUBSECTION---------------------------------------------%
%-------------------------------------------------------------------------------------------------%
\subsection{Reservierte Zeichen}
\begin{frame}[c]
	\begin{center}
		\large Reservierte Zeichen
	\end{center}
\end{frame}
%-----------------------------------------------------------------------------------%
%---------------------------------------FRAME---------------------------------------%
%-----------------------------------------------------------------------------------%
\begin{frame}[fragile]
	\begin{center}
		\begin{tabular}{lll}
			\toprule
			Zeichen			&	Funktion				&	Ersatz			\\ \midrule
			\lstinline|%|	&	Kommentar				&	\lstinline|\%|	\\
			\lstinline|$|	&	Inline-Mathe-Umgebung	&	\lstinline|\$|	\\ 
			\lstinline|#|	&							&	\lstinline|\#|	\\ 
			\lstinline|&|	&	Ausrichtungszeichen		&	\lstinline|\&|	\\ 
			\lstinline|{|	&	Umfasst Argumente		&	\lstinline|\{|	\\ 
			\lstinline|}|	&	Umfasst Argumente		&	\lstinline|\}|	\\ 
			\lstinline|_|	&	Tiefgestellt			&	\lstinline|\_|	\\ 
			\lstinline|^|	&	Hochgestellt			&	\lstinline|\textasciicircum|	\\ 
			\lstinline|~|	&							&	\lstinline|\textasciitilde|		\\ 
			\lstinline|\|	&	Befehl					&	\lstinline|\textbackslash|		\\
			\bottomrule
		\end{tabular}
	\end{center}

	\pause\btVFill
	\Aufgabee
	Kommentiere den Seitenumbruch aus. 
	\vspace{1.3cm}
\note[item]<1>{\# ist für Parameter LaTeX-intern}
\note[item]<1>{\textasciitilde ist geschütztes Leerzeichen}
\end{frame}
%-----------------------------------------------------------------------------------%
%---------------------------------------FRAME---------------------------------------%
%-----------------------------------------------------------------------------------%
\begin{frame}[fragile]
	\Losung
	\begin{outputbox}
		Bei der \textsc{Heisenberg}-Methode lassen sich \textit{Ort} und \textit{Geschwindigkeit} eines bewegten \textbf{Löwen} nicht gleichzeitig bestimmen. Da bewegte \textbf{Löwen} in der Wüste keinen physikalisch sinnvollen \textit{Ort} einnehmen, eignen sie sich auch nicht zur Jagd.
		
		%\newpage
		Die \textbf{Löwenjagd} kann sich demnach zu 100 Prozent auf ruhende \textbf{Löwen} beschränken. Das Fangen eines ruhenden, bewegungslosen \textbf{Löwen} wird dem Leser als Übungsaufgabe überlassen.
	\end{outputbox}

	\Code
	\begin{lstlisting}
Bei der \textsc{Heisenberg}-Methode lassen sich \textit{Ort} und \textit{Geschwindigkeit} eines bewegten \textbf{Löwen} nicht gleichzeitig bestimmen. Da bewegte \textbf{Löwen} in der Wüste keinen physikalisch sinnvollen \textit{Ort} einnehmen, eignen sie sich auch nicht zur Jagd. 

%\newpage
Die \textbf{Löwenjagd} kann sich demnach zu 100 Prozent auf ruhende \textbf{Löwen} beschränken. Das Fangen eines ruhenden, bewegungslosen \textbf{Löwen} wird dem Leser als Übungsaufgabe überlassen.
	\end{lstlisting}
\end{frame}
%-----------------------------------------------------------------------------------%
%---------------------------------------FRAME---------------------------------------%
%-----------------------------------------------------------------------------------%
\begin{frame}[fragile]
	\Aufgabee
	Ersetze das Wort \qquote{Prozent} durch das Prozentzeichen.
	\btVFill\Befehle
	\begin{center}
		\begin{tabular}{lll}
			\toprule
			Zeichen			&	Funktion				&	Ersatz			\\ \midrule
			\lstinline|%|	&	Kommentar				&	\lstinline|\%|	\\
			\lstinline|$|	&	Inline-Mathe-Umgebung	&	\lstinline|\$|	\\ 
			\lstinline|#|	&							&	\lstinline|\#|	\\ 
			\lstinline|&|	&	Ausrichtungszeichen		&	\lstinline|\&|	\\ 
			\lstinline|{|	&	Umfasst Argumente		&	\lstinline|\{|	\\ 
			\lstinline|}|	&	Umfasst Argumente		&	\lstinline|\}|	\\ 
			\lstinline|_|	&	Tiefgestellt			&	\lstinline|\_|	\\ 
			\lstinline|^|	&	Hochgestellt			&	\lstinline|\textasciicircum|	\\ 
			\lstinline|~|	&							&	\lstinline|\textasciitilde|		\\ 
			\lstinline|\|	&	Befehl					&	\lstinline|\textbackslash|		\\
			\bottomrule
		\end{tabular}
	\end{center}
	\vspace{0.1cm}
\end{frame}
%-----------------------------------------------------------------------------------%
%---------------------------------------FRAME---------------------------------------%
%-----------------------------------------------------------------------------------%
\begin{frame}[fragile]
	\Losung
	\begin{outputbox}
		Bei der \textsc{Heisenberg}-Methode lassen sich \textit{Ort} und \textit{Geschwindigkeit} eines bewegten \textbf{Löwen} nicht gleichzeitig bestimmen. Da bewegte \textbf{Löwen} in der Wüste keinen physikalisch sinnvollen \textit{Ort} einnehmen, eignen sie sich auch nicht zur Jagd.
		
		Die \textbf{Löwenjagd} kann sich demnach zu 100 \% auf ruhende \textbf{Löwen} beschränken. Das Fangen eines ruhenden, bewegungslosen \textbf{Löwen} wird dem Leser als Übungsaufgabe überlassen.
	\end{outputbox}

	\Code
	\begin{lstlisting}
Bei der \textsc{Heisenberg}-Methode lassen sich \textit{Ort} und \textit{Geschwindigkeit} eines bewegten \textbf{Löwen} nicht gleichzeitig bestimmen. Da bewegte \textbf{Löwen} in der Wüste keinen physikalisch sinnvollen \textit{Ort} einnehmen, eignen sie sich auch nicht zur Jagd. 

%\newpage
Die \textbf{Löwenjagd} kann sich demnach zu 100 \% auf ruhende \textbf{Löwen} beschränken. Das Fangen eines ruhenden, bewegungslosen \textbf{Löwen} wird dem Leser als Übungsaufgabe überlassen.
	\end{lstlisting}
\end{frame}
%-------------------------------------------------------------------------------------------------%
%------------------------------------------SUBSECTION---------------------------------------------%
%-------------------------------------------------------------------------------------------------%
\subsection{Textstrukturierung}
\begin{frame}[c]
	\begin{center}
		\large Textstrukturierung
	\end{center}
\end{frame}
%-----------------------------------------------------------------------------------%
%---------------------------------------FRAME---------------------------------------%
%-----------------------------------------------------------------------------------%
\begin{frame}[fragile]
	\begin{center}
		\begin{tabular}{lll}
			\toprule
			\LaTeX\ Befehl					&	Level	&	Kommentar	\\ \midrule
			\lstinline|\section{}|			&	2		&	\\
			\lstinline|\subsection{}|		&	3		&	\\
			\lstinline|\subsubsection{}|	&	4		&	\\
			\lstinline|\paragraph{}|		&	5		&	nicht im Inhaltsverzeichnis \\ \midrule
			\lstinline|\textsc{}|			&			&	Kaptitälchen in normalem Text\\
			\lstinline|\textSC{}|			&			&	Kapitälchen in Überschrift\\
			\bottomrule
		\end{tabular}
	\end{center}

	\pause\btVFill
	\Aufgabee
	Gib dem aktuellen Abschnitt die Überschrift
		
		\textrm{\qquote{Die \textsc{Heisenberg}-Methode}}
		
	mit dem Befehl \lstinline[basicstyle=\normalfont\ttfamily\normalsize]|\section| und beachte die Kapitälchen.
	\vspace{1.3cm}
\note[item]<1>{Weitere Level (Part,Chapter,...) verfügbar in anderen Klassen. Kommt später}
\note[item]<2>{Auf die Kaptiälchen hinweisen (textSC)}
\end{frame}
%-----------------------------------------------------------------------------------%
%---------------------------------------FRAME---------------------------------------%
%-----------------------------------------------------------------------------------%
\begin{frame}[fragile]
	\Losung
	\begin{outputbox}
		{ \LARGE\textbf{1 Die \textSC{Heisenberg}-Methode}}
		
		Bei der \textsc{Heisenberg}-Methode lassen sich \textit{Ort} und \textit{Geschwindigkeit} eines bewegten \textbf{Löwen} nicht gleichzeitig bestimmen. Da bewegte \textbf{Löwen}...
	\end{outputbox}

	\Code
	\begin{lstlisting}
\section{Die \textSC{Heisenberg}-Methode}	
	Bei der \textsc{Heisenberg}-Methode lassen sich \textit{Ort} und \textit{Geschwindigkeit} eines bewegten \textbf{Löwen} nicht gleichzeitig bestimmen. Da bewegte \textbf{Löwen} ...
	\end{lstlisting}
\end{frame}
%-----------------------------------------------------------------------------------%
%---------------------------------------FRAME---------------------------------------%
%-----------------------------------------------------------------------------------%
\begin{frame}[fragile]
	\Aufgabee
	Ändere die Überschift \qquote{Die \textsc{Heisenberg}-Methode} in eine \lstinline[basicstyle=\normalfont\ttfamily\normalsize]|\subsubsection| 
	und ergänze darüber
		
		\textrm{\qquote{Methoden der Großwildjagd}}
		
	als \lstinline[basicstyle=\normalfont\ttfamily\normalsize]|\section| und 
		
		\textrm{\qquote{Physikalische Methoden}}
		
	als  \lstinline[basicstyle=\normalfont\ttfamily\normalsize]|\subsection|.
	\btVFill\Befehle
	\begin{center}
		\begin{tabular}{lll}
			\toprule
			\LaTeX\ Befehl					&	Level	&	Kommentar	\\ \midrule
			\lstinline|\section{}|			&	2		&	\\
			\lstinline|\subsection{}|		&	3		&	\\
			\lstinline|\subsubsection{}|	&	4		&	\\
			\lstinline|\paragraph{}|		&	5		&	\\
			\bottomrule
		\end{tabular}
	\end{center}
	\vspace{0.5cm}
	\vspace{0.1cm}
\end{frame}
%-----------------------------------------------------------------------------------%
%---------------------------------------FRAME---------------------------------------%
%-----------------------------------------------------------------------------------%
\begin{frame}[fragile]
	\Losung
	\begin{outputbox}
		{ \LARGE\textbf{1 Methoden der Großwildjagd}} 
		
		{ \Large\textbf{1.1 Physikalische Methoden}}
		
		{ \large\textbf{1.1.1 Die \textSC{Heisenberg}-Methode}}
		
		Bei der \textsc{Heisenberg}-Methode lassen sich \textit{Ort} und \textit{Geschwindigkeit} eines bewegten \textbf{Löwen} nicht gleichzeitig bestimmen. Da bewegte \textbf{Löwen}...
	\end{outputbox}
	
	\Code
	\begin{lstlisting}
\section{Methoden der Großwildjagd}
	\subsection{Physikalische Methoden}
		\subsubsection{Die \textSC{Heisenberg}-Methode}	
			Bei der \textsc{Heisenberg}-Methode lassen sich \textit{Ort} und \textit{Geschwindigkeit} eines bewegten \textbf{Löwen} nicht gleichzeitig bestimmen. Da bewegte \textbf{Löwen} ...
	\end{lstlisting}
\end{frame}
%-----------------------------------------------------------------------------------%
%---------------------------------------FRAME---------------------------------------%
%-----------------------------------------------------------------------------------%
\begin{frame}[fragile]
	\begin{center}
		\begin{tabular}{lll}
			\toprule
			\LaTeX\ Befehl					&	Funktion				\\ \midrule
			\lstinline|\tableofcontents|	&	Inhaltsverzeichnis		\\
			\lstinline|\listoffigures|		&	Abbildungsverzeichnis	\\
			\lstinline|\listoftables| 		&	Tabellenverzeichnis		\\ \bottomrule
		\end{tabular}
	\end{center}

	\pause\btVFill
	\Aufgabee
	Füge nach \lstinline[basicstyle=\normalfont\ttfamily\normalsize]|\begin{document}| und vor der \lstinline[basicstyle=\normalfont\ttfamily\normalsize]|\section| ein Inhaltsverzeichnis mit einem Seitenumbruch ein.
	\vspace{1.3cm}
\end{frame}
%-----------------------------------------------------------------------------------%
%---------------------------------------FRAME---------------------------------------%
%-----------------------------------------------------------------------------------%
\begin{frame}[fragile]
	\Losung
	\begin{outputbox}
		{\large\textbf{Inhaltsverzeichnis}}
		
		{\textbf{1 \hspace{4pt} Methoden der Großwildjagd}} \\
		{\hspace{12pt} 1.1 Physikalische Methoden . . . . . . . . . . . . . . . . . . . . . . . . . . . . . . . . . . \hspace{4pt}  2}
	\end{outputbox}
	\linebreakrule
	\begin{outputbox}
		{ \LARGE\textbf{1 Methoden der Großwildjagd}}
		
		...
	\end{outputbox}

	\Code
	\begin{lstlisting}
\begin{document}
	
\tableofcontents
\newpage 
	
\section{Methoden der Großwildjagd}
	...
	\end{lstlisting}
\end{frame}
%-----------------------------------------------------------------------------------%
%---------------------------------------FRAME---------------------------------------%
%-----------------------------------------------------------------------------------%
\begin{frame}[fragile]
	\Aufgabee
	Neue \lstinline[basicstyle=\normalfont\ttfamily\normalsize]|\subsection| mit dem Titel
		
		\textrm{\qquote{Mathematische Methoden}}
		
	unterhalb des letzten Abschnitts und weiterhin eine neue \lstinline[basicstyle=\normalfont\ttfamily\normalsize]|\subsubsection|
		
		\textrm{\qquote{Die \textSC{Wiener}-\textSC{Tauber}-Methode}}.
	\btVFill\Befehle
	\begin{center}
		\begin{tabular}{lll}
			\toprule
			\LaTeX\ Befehl					&	Level	&	Kommentar	\\ \midrule
			\lstinline|\section{}|			&	2		&	\\
			\lstinline|\subsection{}|		&	3		&	\\
			\lstinline|\subsubsection{}|	&	4		&	\\
			\lstinline|\paragraph{}|		&	5		&	\\
			\bottomrule
		\end{tabular}
	\end{center}
	\vspace{0.1cm}
\end{frame}
%-----------------------------------------------------------------------------------%
%---------------------------------------FRAME---------------------------------------%
%-----------------------------------------------------------------------------------%
\begin{frame}[fragile]
	\Losung
	\begin{outputbox}
		...bewegungslosen \textbf{Löwen} wird dem Leser als Übungsaufgabe überlassen. \\
		\vspace{12pt}
		{ \Large\textbf{2.1. Mathematische Methoden}}
		
		{ \large\textbf{2.1.1. Die \textSC{Wiener}-\textSC{Tauber}-Methode}}
		
	\end{outputbox}

	\Code
	\begin{lstlisting}
...bewegungslosen \textbf{Löwen} wird dem Leser als Übungsaufgabe überlassen.

\subsection{Mathematische Methoden}
	\subsubsection{Die \textSC{Wiener}-\textSC{Tauber}-Methode}
		
\end{document}
	\end{lstlisting}
\note[item]{Benutze zwei getrennte textSC, weil sonst evtl. die Gedankenstriche verschieden aussehen}
\end{frame}