\section{Tabellen}
\begin{frame}[c]
	\begin{center}
		\LARGE \textbf{Tabellen}
	\end{center}
\end{frame}
%%-----------------------------------------------------------------------------------------------%
%%------------------------------------------SUBSECTION-------------------------------------------%
%%-----------------------------------------------------------------------------------------------%
\subsection{calc2LaTeX}
\begin{frame}[c]
	\begin{center}
		\large calc2LaTeX
	\end{center}
\end{frame}
%%-----------------------------------------------------------------------------------%
%%---------------------------------------FRAME---------------------------------------%
%%-----------------------------------------------------------------------------------%
\begin{frame}[c]{Importieren des calc2latex-Makros}
	\begin{onlyenv}<1>
		\begin{figure}[htbp]
			\centering
			\includegraphics[width=0.9\textwidth]{img/Bildschirmfoto_mitKasten/1_Importieren_Macro/1.jpg}
		\end{figure}
	\end{onlyenv}
	\begin{onlyenv}<2>
		\begin{figure}[htbp]
			\centering
			\includegraphics[width=0.9\textwidth]{img/Bildschirmfoto_mitKasten/1_Importieren_Macro/2.jpg}
		\end{figure}
	\end{onlyenv}
	\begin{onlyenv}<3>
		\begin{figure}[htbp]
			\centering
			\includegraphics[width=0.9\textwidth]{img/Bildschirmfoto_mitKasten/1_Importieren_Macro/3.jpg}
		\end{figure}
	\end{onlyenv}
	\begin{onlyenv}<4>
		\begin{figure}[htbp]
			\centering
			\includegraphics[width=0.9\textwidth]{img/Bildschirmfoto_mitKasten/1_Importieren_Macro/4.jpg}
		\end{figure}
	\end{onlyenv}
	\begin{onlyenv}<5>
		\begin{figure}[htbp]
			\centering
			\includegraphics[width=0.9\textwidth]{img/Bildschirmfoto_mitKasten/1_Importieren_Macro/5.jpg}
		\end{figure}
	\end{onlyenv}
	\begin{onlyenv}<6>
		\begin{figure}[htbp]
			\centering
			\includegraphics[width=0.9\textwidth]{img/Bildschirmfoto_mitKasten/1_Importieren_Macro/6.jpg}
		\end{figure}
	\end{onlyenv}
\end{frame}
%%-----------------------------------------------------------------------------------%
%%---------------------------------------FRAME---------------------------------------%
%%-----------------------------------------------------------------------------------%
\begin{frame}[c]{Erstellen einer Tastenkombination für calc2latex}
	\begin{onlyenv}<1>
		\begin{figure}[htbp]
			\centering
			\includegraphics[width=0.9\textwidth]{img/Bildschirmfoto_mitKasten/2_Tastenkombination/1.jpg}
		\end{figure}
	\end{onlyenv}
	\begin{onlyenv}<2>
		\begin{figure}[htbp]
			\centering
			\includegraphics[width=0.9\textwidth]{img/Bildschirmfoto_mitKasten/2_Tastenkombination/2.jpg}
		\end{figure}
	\end{onlyenv}
	\begin{onlyenv}<3>
		\begin{figure}[htbp]
			\centering
			\includegraphics[width=0.9\textwidth]{img/Bildschirmfoto_mitKasten/2_Tastenkombination/3.jpg}
		\end{figure}
	\end{onlyenv}
	\begin{onlyenv}<4>
		\begin{figure}[htbp]
			\centering
			\includegraphics[width=0.9\textwidth]{img/Bildschirmfoto_mitKasten/2_Tastenkombination/4.jpg}
		\end{figure}
	\end{onlyenv}
\end{frame}
%%-----------------------------------------------------------------------------------%
%%---------------------------------------FRAME---------------------------------------%
%%-----------------------------------------------------------------------------------%
\begin{frame}[c]{Erstellen der Tabelle}
	\begin{onlyenv}<1>
		\begin{figure}[htbp]
			\centering
			\includegraphics[width=0.9\textwidth]{img/Bildschirmfoto_mitKasten/3_Tabelle/1.jpg}
		\end{figure}
	\end{onlyenv}
	\begin{onlyenv}<2>
		\begin{figure}[htbp]
			\centering
			\includegraphics[width=0.9\textwidth]{img/Bildschirmfoto_mitKasten/3_Tabelle/2.jpg}
		\end{figure}
	\end{onlyenv}
	\begin{onlyenv}<3>
		\begin{figure}[htbp]
			\centering
			\includegraphics[width=0.9\textwidth]{img/Bildschirmfoto_mitKasten/3_Tabelle/3.jpg}
		\end{figure}
	\end{onlyenv}
	\begin{onlyenv}<4>
		\begin{figure}[htbp]
			\centering
			\includegraphics[width=0.9\textwidth]{img/Bildschirmfoto_mitKasten/3_Tabelle/4.jpg}
		\end{figure}
	\end{onlyenv}
	\begin{onlyenv}<5>
		\begin{figure}[htbp]
			\centering
			\includegraphics[width=0.9\textwidth]{img/Bildschirmfoto_mitKasten/3_Tabelle/5.jpg}
		\end{figure}
	\end{onlyenv}
\end{frame}
%-----------------------------------------------------------------------------------%
%---------------------------------------FRAME---------------------------------------%
%-----------------------------------------------------------------------------------%
\begin{frame}[c]{Ausführen des Makros}
	\begin{onlyenv}<1>
		\begin{figure}[htbp]
			\centering
			\includegraphics[width=0.9\textwidth]{img/Bildschirmfoto_mitKasten/4_Ausfuhren_Macro/1.jpg}
		\end{figure}
	\end{onlyenv}
	\begin{onlyenv}<2>
		\begin{figure}[htbp]
			\centering
			\includegraphics[width=0.9\textwidth]{img/Bildschirmfoto_mitKasten/4_Ausfuhren_Macro/2.jpg}
		\end{figure}
	\end{onlyenv}
	\begin{onlyenv}<3>
		\begin{figure}[htbp]
			\centering
			\includegraphics[width=0.9\textwidth]{img/Bildschirmfoto_mitKasten/3_Tabelle/6.jpg}
		\end{figure}
	\end{onlyenv}
	\begin{onlyenv}<4>
		\begin{figure}[htbp]
			\centering
			\includegraphics[width=0.9\textwidth]{img/Bildschirmfoto_mitKasten/3_Tabelle/7.jpg}
		\end{figure}
	\end{onlyenv}
\end{frame}
%%-----------------------------------------------------------------------------------%
%%---------------------------------------FRAME---------------------------------------%
%%-----------------------------------------------------------------------------------%
\begin{frame}[fragile]
	\Aufgabee
		Füge die erstellte Tabelle als Inhalt von einer neuen, ersten \lstinline[basicstyle=\normalfont\ttfamily\normalsize]|\section|
        
        \textrm{\qquote{Vergleich der Körpereigenschaften}}
        
        ein.
	\begin{outputbox}
		{\LARGE \textbf{1 Vergleich der Körpereigenschaften}}
		\vspace{-0.3cm}
		\begin{center}
            \begin{table}[htbp] 
                \centering
            	\caption{Vergleich von Löwe und Mensch}
                \vspace{-0.4cm}
            	\begin{tabular}{|r|c|c|}
            		\hline 
            		\textbf{}                         & \textbf{Löwe} & \textbf{Mensch}  \\ \hline 
            		\textbf{Gewicht des Gehirns}      & 1             & 2                \\ \hline 
            		\textbf{Körpertemperatur}         & 3             & 4                \\ \hline 
            		\textbf{Atemfrequenz}             & 5             & 6                \\ \hline 
            		\textbf{Maximale Geschwindigkeit} & 7             & 8                \\ \hline 
            		\textbf{Grundstoffwechselumsatz}  & 9             & 10               \\ \hline
            	\end{tabular} 
            	\label{}
            \end{table}
		\end{center}
	\end{outputbox}

%	\btVFill\Befehle
%	\begin{center}
%		\begin{tabular}{ll}
%			\toprule
%			\LaTeX\ Befehl								&	Funktion								\\ \midrule
%			\lstinline|\begin{table}...\end{table}|		&	Umgebung für Tabellen					\\
%			\lstinline|\begin{tabular}...\end{tabular}|	&	Erstellt eine Tabelle					\\ 
%			\lstinline|&|								&	Sprung zur nächsten Zelle				\\
%			\lstinline|\\|								&	Neue Zeile								\\
%			\bottomrule
%		\end{tabular}
%	\end{center}
%	\vspace{0.1cm}
\end{frame}
%%-----------------------------------------------------------------------------------%
%%---------------------------------------FRAME---------------------------------------%
%%-----------------------------------------------------------------------------------%
\begin{frame}[fragile]
    \vspace{-0.2cm}
    \Losung
    \vspace{-0.1cm}
	\begin{outputbox}
        \vspace{-0.4cm}
		\begin{center}
            \begin{table}[htbp] 
                \centering
                \caption{Vergleich von Löwe und Mensch}
                \vspace{-0.4cm}
                \begin{tabular}{|r|c|c|}
                    \hline 
                    \textbf{}                         & \textbf{Löwe} & \textbf{Mensch}  \\ \hline 
                    \textbf{Gewicht des Gehirns}      & 1             & 2                \\ \hline 
                    \textbf{Körpertemperatur}         & 3             & 4                \\ \hline 
                    \textbf{Atemfrequenz}             & 5             & 6                \\ \hline 
                    \textbf{Maximale Geschwindigkeit} & 7             & 8                \\ \hline 
                    \textbf{Grundstoffwechselumsatz}  & 9             & 10               \\ \hline
                \end{tabular} 
                \label{}
            \end{table}
		\end{center}
        \vspace{-0.4cm}
	\end{outputbox}
    \vspace{-0.2cm}
    \Code
    \vspace{-0.1cm}
	\begin{lstlisting}[gobble=8]
        \begin{table}[htbp]
          \centering
          \caption{Vergleich von Löwe und Mensch}
          \begin{tabular}{|r|c|c|}
            \hline 
            \textbf{}                    & \textbf{Löwe} & \textbf{Mensch}  \\ \hline 
            \textbf{Gewicht des Gehirns}      & 1             & 2           \\ \hline 
            \textbf{Körpertemperatur}         & 3             & 4           \\ \hline 
            \textbf{Atemfrequenz}             & 5             & 6           \\ \hline 
            \textbf{Maximale Geschwindigkeit} & 7             & 8           \\ \hline 
            \textbf{Grundstoffwechselumsatz}  & 9             & 10          \\ \hline
          \end{tabular} 
          \label{}
        \end{table}
	\end{lstlisting}
\end{frame}
%%-----------------------------------------------------------------------------------%
%%---------------------------------------FRAME---------------------------------------%
%%-----------------------------------------------------------------------------------%
\begin{frame}[fragile]
	\Aufgabee
		Füge die folgenden Trennstriche in die Tabelle ein und gib der Tabelle ein \textbf{Über}schrift
        
        \textrm{\qquote{Vergleich von Löwe und Mensch.}}.
	\begin{outputbox}
		\vspace{-0.4cm}
		\begin{center}
            \begin{table}[htbp]
                \centering
            	\caption{Vergleich von Löwe und Mensch.}
                \vspace{-0.4cm}
            	\begin{tabular}{r|cc}
            		\hline 
            		\textbf{} & \textbf{Löwe} & \textbf{Mensch} \\ \hline  
            		\textbf{Gewicht des Gehirns} & 1 & 2 \\  
            		\textbf{Körpertemperatur} & 3 & 4 \\  
            		\textbf{Atemfrequenz} & 5 & 6 \\ 
            		\textbf{Maximale Geschwindigkeit} & 7 & 8 \\  
            		\textbf{Grundstoffwechselumsatz} & 9 & 10 \\ \hline
            	\end{tabular} 
            	\label{}
            \end{table}
		\end{center}
		\vspace{-0.2cm}
	\end{outputbox}
	
	\btVFill\Befehle
	\begin{center}
		\begin{tabular}{ll}
			\toprule
			\LaTeX\ Befehl								&	Funktion								\\ \midrule
			\lstinline|\hline|							&	Horizontaler Trennstrich				\\
			\lstinline/{|Spalte|Spalte|}/				&	Vertikaler Trennstrich    				\\
			\bottomrule
		\end{tabular}
	\end{center}
	\vspace{0.1cm}
\end{frame}
%%-----------------------------------------------------------------------------------%
%%---------------------------------------FRAME---------------------------------------%
%%-----------------------------------------------------------------------------------%
\begin{frame}[fragile]
	\Code
	\begin{lstlisting}[gobble=8]
        \begin{table}[htbp]
          \centering
          \caption{Vergleich von Löwe und Mensch}
          \begin{tabular}{|r|c|c|}
            \hline 
            \textbf{}                    & \textbf{Löwe} & \textbf{Mensch}  \\ \hline 
            \textbf{Gewicht des Gehirns}      & 1             & 2           \\
            \textbf{Körpertemperatur}         & 3             & 4           \\
            \textbf{Atemfrequenz}             & 5             & 6           \\
            \textbf{Maximale Geschwindigkeit} & 7             & 8           \\ 
            \textbf{Grundstoffwechselumsatz}  & 9             & 10          \\ \hline 
          \end{tabular} 
          \label{}
        \end{table}
	\end{lstlisting}
\end{frame}
%%%-----------------------------------------------------------------------------------------------%
%%%------------------------------------------SUBSECTION-------------------------------------------%
%%%-----------------------------------------------------------------------------------------------%
%\subsection{Tabellen- und Abbildungsverzeichnisse}
%\begin{frame}[c]
%	\begin{center}
%		\large Tabellen- und Abbildungsverzeichnisse
%	\end{center}
%\end{frame}
%\note{
%- Vor allem in längeren Dokumenten praktisch\\
%- beim AP vielleicht eher nicht notwendig (?)}
%%-----------------------------------------------------------------------------------%
%%---------------------------------------FRAME---------------------------------------%
%%-----------------------------------------------------------------------------------%
%\begin{frame}[fragile]
%	\begin{center}
%		\begin{tabular}{lp{8cm}}
%			\toprule
%			\LaTeX\ Befehl					&	Funktion								\\ \midrule
%			\lstinline|\listoffigures|		&	Erstellt ein Verzeichnis aller Abbildungen im Dokument\newline(genauer: aller \lstinline[basicstyle=\normalsize\normalfont]|\begin{figure}...\end{figure}| Umgebungen)		\\
%			\lstinline|\listoftables|		&	Selbe Funktionalität, bloß für \lstinline[basicstyle=\normalsize\normalfont]|\begin{table}...\end{table}|			\\
%			\bottomrule
%		\end{tabular}
%	\end{center}
	
%	\pause\btVFill
%	\begin{Aufgabe}
%		Füge am Ende des Dokuments ein Tabellen- und ein Abbildungsverzeichnis ein.
%	\end{Aufgabe}
%	\vspace{0.3cm}
%\note<1>{
%- kann nur die Verzeichnisse erstellen, welche sich in den entsprechenden Umgebungen befinden - das ist der grund für \textrm{begin table ... end table} und \textrm{begin figure ... end figure}}
%\end{frame}
%-----------------------------------------------------------------------------------%
%---------------------------------------FRAME---------------------------------------%
%-----------------------------------------------------------------------------------%
%\begin{frame}[fragile]
%	\Code
%	\begin{lstlisting}
%\listoffigures
%\listoftables
%	\end{lstlisting}
%\end{frame}