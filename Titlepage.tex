\documentclass["WS\space 16-17\space -\space LaTeX-Kurs\space -\space Praesentation\space -\space 3.tex"]{subfiles}

\begin{document}

\section{Titelseiten}
\begin{frame}[c]
	\begin{center}
		\LARGE \textbf{Titelseiten}
	\end{center}
\end{frame}
%%-----------------------------------------------------------------------------------------------%
%%------------------------------------------SUBSECTION-------------------------------------------%
%%-----------------------------------------------------------------------------------------------%
\subsection*{Titelseiten}
%\begin{frame}
%	\begin{center}
%		\large Grundlagen
%	\end{center}
%\end{frame}
%-----------------------------------------------------------------------------------%
%---------------------------------------FRAME---------------------------------------%
%-----------------------------------------------------------------------------------%
\begin{frame}[fragile]
	Befehle, die für das einfache Erstellen von Titelseiten gedacht sind:
	\begin{itemize}
		\item[-]<1->
			\lstinline|\title{Titel}|
		\item[-]<2->
			\lstinline|\author{Autor}|
		\item[-]<3->
			\lstinline|\date{Datum}|
		\item[-]<4->
			\lstinline|\maketitle|
	\end{itemize}

	\Code
	\begin{lstlisting}
\title{\LaTeX\ Kurs}
\author{Eure Tutoren}
\date{\today}
\maketitle
	\end{lstlisting}
\end{frame}
\note{\large- Es gibt auch noch eine Titlepage-Umgebung, wenn man MEHR will - das hier ist nur das Minimalbeispiel und sollte für AP vollkommen ausreichend sein\\- weitere Titlepages evtl am 4. Termin}
%-----------------------------------------------------------------------------------%
%---------------------------------------FRAME---------------------------------------%
%-----------------------------------------------------------------------------------%
\begin{frame}[fragile]
	\begin{Aufgabe}
		Erstelle eine Titelseite:
		\begin{center}
			\begin{tabular}{ll}
				Titel	&	\qquote{Das Fangen von Löwen}													\\
				Autor	&	dein Name																		\\
				Datum	&	\qquote{Konstanz, den XX} (setze das heutige Datum mit \lstinline[basicstyle=\normalfont\normalsize]|\today|)	\\
			\end{tabular}
		\end{center}
	\end{Aufgabe}
	\btVFill\Befehle
	\begin{center}
		\begin{tabular}{ll}
			\toprule
			\LaTeX\ Befehl									&	Funktion					\\ \midrule
			\lstinline|\title{}|	&	\\
			\lstinline|\author{}|	&	\\
			\lstinline|\date{}|		&	\\
			\lstinline|\today|		&	\\
			\lstinline|\maketitle|	&	\\
			\bottomrule
		\end{tabular}
	\end{center}
	\vspace{0.1cm}
\end{frame}
%-----------------------------------------------------------------------------------%
%---------------------------------------FRAME---------------------------------------%
%-----------------------------------------------------------------------------------%
\begin{frame}[fragile]
	\Losung
	\begin{outputbox}
		\begin{center}
			\textsf{{\LARGE Das Fangen von Löwen}
			\\\vspace{0.5cm}
			Teil N. Ehmer
			\\\vspace{0.2cm}
			Konstanz, den XX. November 2016}
		\end{center}
	\end{outputbox}
		
	\Code
	\begin{lstlisting}
\title{Das Fangen von Löwen}
\author{Teil N. Ehmer}
\date{Konstanz, den \today}
\maketitle
	\end{lstlisting}
\end{frame}

\end{document}