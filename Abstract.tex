\section{Abstract}
\begin{frame}[c]
	\begin{center}
		\LARGE \textbf{Abstract}
	\end{center}
\end{frame}
%%-----------------------------------------------------------------------------------------------%
%%------------------------------------------SUBSECTION-------------------------------------------%
%%-----------------------------------------------------------------------------------------------%
\subsection*{Abstract}
%\begin{frame}
%	\begin{center}
%		\large Grundlagen
%	\end{center}
%\end{frame}
%-----------------------------------------------------------------------------------%
%---------------------------------------FRAME---------------------------------------%
%-----------------------------------------------------------------------------------%
\begin{frame}[fragile]
	\vspace{-0.3cm}
	\begin{Aufgabe}
		Füge am Anfang des Dokuments (nach dem \lstinline[basicstyle=\normalfont\normalsize]|\maketitle| Befehl) eine Zusammenfassung mit Hilfe der \lstinline[basicstyle=\normalfont\normalsize]|\begin{abstract}...\end{abstract}|-Umgebung hinzu.
		
		Schreibe darin folgenden Text und benutze die  Umgebung und verwende für die Angaben von Titel, Autor und Jahr die Befehle \lstinline[basicstyle=\normalfont\normalsize]|\citetitle|, \lstinline[basicstyle=\normalfont\normalsize]|\citeauthor| und \lstinline[basicstyle=\normalfont\normalsize]|\citeyear|:
	\end{Aufgabe}
	\begin{outputbox}
		Dies ist das Endprodukt des \LaTeX-Kurses der Fachschaft Physik der Universität Konstanz. Der Text basiert auf der Veröffentlichung \citetitle{Petard1938} von \citeauthor{Petard1938} aus dem Jahr \citeyear{Petard1938} \cite[29\psqq]{Weber2013} und einem Dokument von XXX, das online zu finden ist. Der Text weicht an einigen Stellen von den Originalen ab, wenn die Pädagogik es verlangte.
	\end{outputbox}
	\btVFill\Befehle
	\begin{center}
		\begin{tabular}{ll}
			\toprule
			\LaTeX\ Befehl							&	Funktion					\\ \midrule
			\lstinline|\cite[]{}|					&	Zitierbefehl\\
			\lstinline|\printbibliography|			&	Erstellen des Literaturverzeichnis\\
			\lstinline|\citeauthor{}|				&	Zitieren des Autors \\
			\lstinline|\citeyear{}|					&	Zitieren des Jahrs \\
			\lstinline|\citetitle{}|				&	Zitieren des Titels \\
			\lstinline|\psq|, \lstinline|\psqq|		&	\texttt{f} und \texttt{ff} \\
			\bottomrule
		\end{tabular}
	\end{center}
	\vspace{0.1cm}
\end{frame}
%-----------------------------------------------------------------------------------%
%---------------------------------------FRAME---------------------------------------%
%-----------------------------------------------------------------------------------%
\begin{frame}[fragile]
	\Losung
	\begin{outputbox}
		Dies ist das Endprodukt des \LaTeX-Kurses der Fachschaft Physik der Universität Konstanz. Der Text basiert auf der Veröffentlichung \citetitle{Petard1938} von \citeauthor{Petard1938} aus dem Jahr \citeyear{Petard1938} \cite[29\psqq]{Weber2013} und einem Dokument von XXX, das online zu finden ist. Der Text weicht an einigen Stellen von den Originalen ab, wenn die Pädagogik es verlangte.
	\end{outputbox}

	\Code
	\begin{lstlisting}
Dies ist das Endprodukt des \LaTeX-Kurses der Fachschaft Physik der Universität Konstanz. Der Text basiert auf der Veröffentlichung \citetitle{Petard1938} von \citeauthor{Petard1938} aus dem Jahr \citeyear{Petard1938} \cite[29\psqq]{Weber2013} und einem Dokument von XXX, das online zu finden ist. Der Text weicht an einigen Stellen von den Originalen ab, wenn die Pädagogik es verlangte.	
	\end{lstlisting}
\end{frame}