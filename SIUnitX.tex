\section{SIUnitX}
\begin{frame}[c]
	\begin{center}
		\LARGE \textbf{SIUnitX}
	\end{center}
\end{frame}
%%-----------------------------------------------------------------------------------------------%
%%------------------------------------------SUBSECTION-------------------------------------------%
%%-----------------------------------------------------------------------------------------------%
\subsection{Grundlagen}
\begin{frame}[c]
	\begin{center}
		\large Grundlagen
	\end{center}
\end{frame}


%-----------------------------------------------------------------------------------%
%---------------------------------------FRAME---------------------------------------%
%-----------------------------------------------------------------------------------%
\begin{frame}[fragile]
		Das SIUnitX-Paket bietet eine korrekte Darstellung von SI-Einheiten:
	\begin{center}
	\begin{tabular}{r|ll|l}
		\toprule
		\LaTeX\ Befehl					& Beispiel		&Ausgabe	&	Funktion								\\ \midrule
		\lstinline|\SI{}{}|			&	\lstinline|\SI{220}{\gram}| & \SI{220}{\gram}			&	Zahl + Einheit				\\
		\lstinline/\si{}/			&	\lstinline|\si{\gram}|	& \si{\gram}				&	\underline{nur} Einheit					\\
			\lstinline/\num{}/			&	\lstinline|\num{220}|& \num{220}					&	\underline{nur} Zahl					\\
		\bottomrule
	\end{tabular}
\end{center}
Einheiten können mit Abkürzungen oder ausgeschrieben angegeben werden. Allerdings muss mit einem {\textbackslash} eingeleitet werden:
	\begin{center}
	\begin{tabular}{ll|c}
	\toprule
	Ausgeschrieben						& Abgekürzt						& Ausgabe				\\ \midrule
	\lstinline/\si{\gram}/				&	\lstinline|\si{\g}|			& \si{\gram}			\\
	\lstinline/\si{\degreeCelsius}/		&	\lstinline|\si{\celsius}|	& \si{\celsius}		\\
	\lstinline/\si{\watt}/				&	\lstinline|\si{\W}|			& \si{\W}			\\
	\bottomrule
\end{tabular}
\end{center}
Ausgeschrieben ist die Einheit klein zuschreiben, wird sie abgekürzt muss sie entsprechend der Notationgroß oder klein geschrieben werden (siehe \lstinline|\si{\g}| - \si{\g}, \lstinline|\si{\W}| - \si{\W}).
\end{frame}



%-----------------------------------------------------------------------------------%
%---------------------------------------FRAME---------------------------------------%
%-----------------------------------------------------------------------------------%
\begin{frame}[fragile]
	\vspace{-0.3cm}
	\begin{Aufgabe}
	Ergänze die Tabelle um folgende Angaben mit Hilfe der entsprechenden Befehle:
	\end{Aufgabe}
	\begin{outputbox}
		\begin{center}
			\begin{tabular}{r|cc}
				\hline
				&	\textbf{Löwe}										& \textbf{Mensch} 						\\ \hline
				\textbf{Gewicht des Gehirns}		&	\SI{220}{\g}										& 2			\\ 
				\textbf{Körpertemperatur}			&	\SI{37.9}{\degreeCelsius}							& 3	\\
				\textbf{Atemfrequenz}				&	5								& 6 					\\
				\textbf{Maximale Geschwindigkeit}	&	7								& 8				\\ 
				\textbf{Grundstoffwechselumsatz}	& 9	&  \SI{82.78}{\watt}\\
				\hline
			\end{tabular}
		\end{center}
	\end{outputbox}
	
	\begin{center}
		\begin{tabular}{ll}
			\toprule
			\LaTeX\ Befehl						&	Funktion						\\ \midrule
			\lstinline|\SI{}{}|					&	Zahl + Einheit					\\
			\lstinline|\einheit|				&	ausgeschriebene Einheit			\\
			\lstinline|\E| oder \lstinline|\e|	&	abgekürzte Einheit				\\
			\bottomrule
		\end{tabular}
	\end{center}
	\vspace{0.1cm}
\end{frame}

%-----------------------------------------------------------------------------------%
%---------------------------------------FRAME---------------------------------------%
%-----------------------------------------------------------------------------------%
\begin{frame}[fragile]
\Losung
	\begin{outputbox}
		\begin{center}
			\begin{tabular}{r|cc}
				\hline
				&	\textbf{Löwe}										& \textbf{Mensch} 						\\ \hline
				\textbf{Gewicht des Gehirns}		&	\SI{220}{\g}										& 2			\\ 
				\textbf{Körpertemperatur}			&	\SI{37.9}{\celsius}							& 3	\\
				\textbf{Atemfrequenz}				&	5								& 6 					\\
				\textbf{Maximale Geschwindigkeit}	&	7								& 8				\\ 
				\textbf{Grundstoffwechselumsatz}	& 9	&  \SI{82.78}{\watt}\\
				\hline
			\end{tabular}
		\end{center}
	\end{outputbox}
			\begin{center}
			\begin{tabular}{ll}
				\toprule
				\LaTeX\ Befehl						&	Ausgabe						\\ \midrule
				\lstinline|\SI{220}{\g}|			&	\SI{220}{\g}					\\
				\lstinline|\SI{37.9}{\celsius}|		&	\SI{37.9}{\celsius}			\\
				\lstinline|\SI{82.78}{\watt}|		&	\SI{82.78}{\watt}			\\
				\bottomrule
			\end{tabular}
		\end{center}
	
	
	
\end{frame}
%-----------------------------------------------------------------------------------%
%---------------------------------------FRAME---------------------------------------%
%-----------------------------------------------------------------------------------%
\begin{frame}[fragile]
	\begin{center}
		\begin{tabular}{r|ll|l}
			\toprule
			\LaTeX\ Befehl				& Beispiel		&Ausgabe	&	Funktion								\\ \midrule
			\lstinline|\SIrange{}{}{}|	&	\lstinline|\SIrange{35.9}{38}{\celsius}| & \SIrange{35.9}{38}{\celsius}			&	Zahl bis Zahl + Einheit				\\
			\lstinline/\pm/			&	\lstinline|\num{1.35 \pm 0.15}|	& \num{1.35 \pm 0.15}				&	 Unsicherheit					\\
			\lstinline/e/			&	\lstinline|\num{2.2 e2}|& \num{2.2 e2}					&	 Zehnerpotenzen					\\
			\bottomrule
		\end{tabular}
	\end{center}
		Größenordnungen können wie Einheiten angegeben werden. Allerdings muss mit einem {\textbackslash} eingeleitet werden:
	\begin{center}
		\begin{tabular}{ll|c}
			\toprule
			Ausgeschrieben						& Abgekürzt						& Ausgabe				\\ \midrule
			\lstinline/\si{\gram}/				&	\lstinline|\si{\g}|			& \si{\gram}			\\
			\lstinline/\si{\milli\gram}/				&	\lstinline|\si{\mg}|			& \si{\mg}			\\
			\lstinline/\si{\kilo\gram}/				&	\lstinline|\si{\kg}|			& \si{\kg} 		\\
			\bottomrule
		\end{tabular}
	\end{center}
Einige Regeln: 
\begin{itemize}
	\item Kombinationen von Abkürzung/Ausgeschieben wie \lstinline|\si{\mgram}| funktionieren nicht
	\item Ungewöhnliche Kombinationen wie \lstinline|\si{\mega\gram}| müssen ausgeschrieben formuliert werden 
\end{itemize}

\end{frame}
%-----------------------------------------------------------------------------------%
%---------------------------------------FRAME---------------------------------------%
%-----------------------------------------------------------------------------------%
\begin{frame}[fragile]
	\vspace{-0.3cm}
	\begin{Aufgabe}
		Ergänze die Tabelle um folgende Angaben mit Hilfe der entsprechenden Befehle:
	\end{Aufgabe}
	\begin{outputbox}
		\begin{center}
			\begin{tabular}{r|cc}
				\hline
				&	\textbf{Löwe}										& \textbf{Mensch} 						\\ \hline 
				\textbf{Gewicht des Gehirns}		&	\SI{2.2 e2}{\g}										& \SI{1.35 \pm 0.15}{\kg}				\\ 
				\textbf{Körpertemperatur}			&	\SI{37.9}{\degreeCelsius}							& \SIrange{35.9}{38}{\degreeCelsius}	\\
				\textbf{Atemfrequenz}				&	5								& 6					\\
				\textbf{Maximale Geschwindigkeit}	&	7								& 8				\\ 
				\textbf{Grundstoffwechselumsatz}	&	9	&  \SI{82.78}{\watt}\\
				\hline
			\end{tabular}
		\end{center}
	\end{outputbox}

	\begin{center}
	\begin{tabular}{ll}
		\toprule
		\LaTeX\ Befehl						&	Funktion									\\ \midrule
		\lstinline|\SIrange{}{}|			&	Zahl bis Zahl + Einheit						\\
		\lstinline|\pm			|			&	Unsicherheit								\\		
		\lstinline|e			|			&	Zehnerpotenzen						\\		
		\lstinline|\größenordnung\einheit|	&	ausgeschriebene Einheit	mit Größenordnung	\\
		\lstinline|\gE|, \lstinline|\ge|, \lstinline|\Ge| oder \lstinline|\GE|&	abgekürzte Einheit mit Größenordnung		\\
		\bottomrule
	\end{tabular}
\end{center}
	\vspace{0.1cm}
\end{frame}

%-----------------------------------------------------------------------------------%
%---------------------------------------FRAME---------------------------------------%
%-----------------------------------------------------------------------------------%
\begin{frame}[fragile]
	\Losung
	\begin{outputbox}
		\begin{center}
			\begin{tabular}{r|cc}
				\hline
				&	\textbf{Löwe}										& \textbf{Mensch} 						\\ \hline
				\textbf{Gewicht des Gehirns}		&	\SI{2.2 e2}{\g}										& \SI{1.35 \pm 0.15}{\kg}			\\ 
				\textbf{Körpertemperatur}			&	\SI{37.9}{\celsius}							& \SIrange{35.9}{38}{\degreeCelsius}	\\
				\textbf{Atemfrequenz}				&	5								& 6 					\\
				\textbf{Maximale Geschwindigkeit}	&	7								& 8				\\ 
				\textbf{Grundstoffwechselumsatz}	& 9	&  \SI{82.78}{\watt}\\
				\hline
			\end{tabular}
		\end{center}
	\end{outputbox}

				\begin{center}
		\begin{tabular}{ll}
			\toprule
			\LaTeX\ Befehl						&	Ausgabe						\\ \midrule
			\lstinline|\SI{2.2 e2}{\g} |			&	\SI{2.2 e2}{\g} 					\\
			\lstinline|\SI{1.35 \pm 0.15}{\kg}|		&	\SI{1.35 \pm 0.15}{\kg}			\\
			\lstinline|\SIrange{35.9}{38}{\celsius}|		&	\SIrange{35.9}{38}{\celsius}			\\
			\bottomrule
		\end{tabular}
	\end{center}
	
	
\end{frame}

%-----------------------------------------------------------------------------------%
%---------------------------------------FRAME---------------------------------------%
%-----------------------------------------------------------------------------------%
\begin{frame}[fragile]
	\begin{center}
		\begin{tabular}{r|ll|l}
			\toprule
			\LaTeX\ Befehl				& Beispiel		&Ausgabe	&	Funktion								\\ \midrule
			\lstinline|\squared|	&	\lstinline|\si{\meter\squared}| &\si{\meter\squared}			&	Quadrat 				\\
						\lstinline|\cubed|	&	\lstinline|\si{\meter\cubed}| &\si{\meter\cubed}			&	hoch drei				\\
			\lstinline/\tothe{}/			&	\lstinline|\si{\meter\tothe{4}}|	& \si{\meter\tothe{2}}				&	 positive Exponenten					\\
			\lstinline/\per/			&	\lstinline|\si{\per\meter}|& \si{\per\meter}				&	 	Bruch			\\
			\bottomrule
		\end{tabular}
	\end{center}
Jede Einheit muss extra mit einem Exponenten versehen werden. Dies gilt auch für Einheiten die unterhalb des Bruchstrichs stehen sollen:
	\vspace{12px}
\begin{columns}[c]
\column{0.6\linewidth}
\begin{lstlisting}
\si{\meter\tothe{1}\meter\squared\meter\cubed
\meter\tothe{4}}

\si{\per\meter\per\meter\per\meter}

\si{\per\meter\tothe{1}\per\meter\squared\per
\meter\cubed\per\meter\tothe{4}}
\end{lstlisting}
\column{0.4\linewidth}
\begin{outputbox}
	\begin{center}
		\vspace{3px}
\si{\meter\tothe{1}\meter\squared\meter\cubed\meter\tothe{4}} \\
\vspace{10px}
\si{\per\meter\per\meter\per\meter} \\
\vspace{10px}
\si{\per\meter\tothe{1}\per\meter\squared\per\meter\cubed\per\meter\tothe{4}}
	\end{center}
\end{outputbox}
\end{columns}
\end{frame}

%-----------------------------------------------------------------------------------%
%---------------------------------------FRAME---------------------------------------%
%-----------------------------------------------------------------------------------%
\begin{frame}[fragile]
	\begin{Aufgabe}
	Ergänze die Tabelle um folgende Angaben mit Hilfe der entsprechenden Befehle:
\end{Aufgabe}
	\begin{outputbox}
		\vspace{-0.1cm}
		\begin{center}
			\begin{tabular}{r|cc}
				\hline
				&	\textbf{Löwe}										& \textbf{Mensch} 						\\ \hline
				\textbf{Gewicht des Gehirns}		&	\SI{2.2e2}{\g}										& \SI{1.35 \pm 0.15}{\kg}				\\ 
				\textbf{Körpertemperatur}			&	\SI{37.9}{\degreeCelsius}							& \SIrange{35.9}{38}{\degreeCelsius}	\\
				\textbf{Atemfrequenz}				&	\SI{10}{\per \minute}								& \SI{15840}{\per\day} 					\\
				\textbf{Maximale Geschwindigkeit}	&	\SI{20.83}{\m\per\s}								& \SI{44.46}{\km\per\hour} 				\\ 
				\textbf{Grundstoffwechselumsatz}	&	\SI{94.58}{\kg\meter\squared\per\second\tothe{3}}	&  \SI{82.78}{\watt}\\
				\hline
			\end{tabular}
		\end{center}
		\vspace{-0.1cm}
	\end{outputbox}
	
	\begin{center}
		\begin{tabular}{ll}
	\toprule
	\LaTeX\ Befehl				& 	Funktion								\\ \midrule
	\lstinline|\squared|	&	Quadrat 				\\
	\lstinline|\cubed|				&	hoch drei				\\
	\lstinline/\tothe{}/			&	 positive Exponenten					\\
	\lstinline/\per/		&	Bruch			\\
	\bottomrule
\end{tabular}
	\end{center}
\end{frame}


%-----------------------------------------------------------------------------------%
%---------------------------------------FRAME---------------------------------------%
%-----------------------------------------------------------------------------------%
\begin{frame}[fragile]
	\Losung
	\begin{outputbox}
		\vspace{-0.1cm}
		\begin{center}
			\begin{tabular}{r|cc}
				\hline
				&	\textbf{Löwe}										& \textbf{Mensch} 						\\ \hline
				\textbf{Gewicht des Gehirns}		&	\SI{2.2e2}{\g}										& \SI{1.35 \pm 0.15}{\kg}				\\ 
				\textbf{Körpertemperatur}			&	\SI{37.9}{\degreeCelsius}							& \SIrange{35.9}{38}{\degreeCelsius}	\\
				\textbf{Atemfrequenz}				&	\SI{10}{\per \minute}								& \SI{15840}{\per\day} 					\\
				\textbf{Maximale Geschwindigkeit}	&	\SI{20.83}{\m\per\s}								& \SI{44.46}{\km\per\hour} 				\\ 
				\textbf{Grundstoffwechselumsatz}	&	\SI{94.58}{\kg\meter\squared\per\second\tothe{3}}	&  \SI{82.78}{\watt}\\
				\hline
			\end{tabular}
		\end{center}
		\vspace{-0.1cm}
	\end{outputbox}

				\begin{center}
	\begin{tabular}{ll}
		\toprule
		\LaTeX\ Befehl						&	Ausgabe						\\ \midrule
		\lstinline|\SI{10}{\per\minute} |			&	\SI{10}{\per\minute} 					\\
		\lstinline|\SI{15840}{\per\day} |		&	\SI{15840}{\per\day} 		\\
		\lstinline|\SI{20.83}{\m\per\s} |		&	\SI{20.83}{\m\per\s}			\\
		\lstinline|\SI{44.46}{\km\per\hour} |		&	\SI{44.46}{\km\per\hour}			\\
		\lstinline|\SI{94.58}{\kg\m\squared\per\s\tothe{3}} |		&	\SI{94.58}{\kg\m\squared\per\s\tothe{3}}			\\
		\bottomrule
	\end{tabular}
\end{center}
\end{frame}
