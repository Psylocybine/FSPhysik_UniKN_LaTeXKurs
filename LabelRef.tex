\documentclass["WS\space 16-17\space -\space LaTeX-Kurs\space -\space Praesentation\space -\space 3.tex"]{subfiles}

\begin{document}

\section{Referenzen innerhalb des Dokuments}
\begin{frame}[c]
	\begin{center}
		\LARGE \textbf{Referenzen innerhalb des Dokuments}
	\end{center}
\end{frame}
%%-----------------------------------------------------------------------------------------------%
%%------------------------------------------SUBSECTION-------------------------------------------%
%%-----------------------------------------------------------------------------------------------%
\subsection*{Referenzen innerhalb des Dokuments}
%\begin{frame}
%	\begin{center}
%		\large Grundlagen
%	\end{center}
%\end{frame}
%-----------------------------------------------------------------------------------%
%---------------------------------------FRAME---------------------------------------%
%-----------------------------------------------------------------------------------%
\begin{frame}[fragile]
	\Ausgabe
	\begin{outputbox}
		\begin{align}
			a^2+b^2=c^2\label{eq:satzpythagoras}
		\end{align}
		Gleichung \eqref{eq:satzpythagoras} stellt einen sehr wichtigen mathematischer Ausdruck dar.
	\end{outputbox}
	
	
	\pause
	\begin{lstlisting}
\label{NAME}
	\end{lstlisting}
	
	\pause
	\Code
	\begin{onlyenv}<3>
		\begin{lstlisting}
\begin{align}
	a^2+b^2=c^2
\end{align}
		\end{lstlisting}
	\end{onlyenv}
	\begin{onlyenv}<4>
		\begin{lstlisting}
\begin{align}
	a^2+b^2=c^2\label{eq:satzpythagoras}
\end{align}
		\end{lstlisting}
	\end{onlyenv}
	\begin{onlyenv}<5>
		\begin{lstlisting}
\begin{align}
		a^2+b^2=c^2\label{eq:satzpythagoras}
\end{align}
Gleichung \eqref{eq:satzpythagoras} stellt einen sehr wichtigen mathematischer Ausdruck dar.
		\end{lstlisting}
	\end{onlyenv}
\end{frame}
%-----------------------------------------------------------------------------------%
%---------------------------------------FRAME---------------------------------------%
%-----------------------------------------------------------------------------------%
\begin{frame}[fragile]
	\Code
	\begin{lstlisting}
\begin{figure}
	\centering
	\includegraphics[]{...}
	\caption{Testabbildung mit interner Referenz}
	\label{fig:abb1}
\end{figure}	
Abbildung \ref{fig:abb1} zeigt nichts.		
	\end{lstlisting}

	\Ausgabe
	\begin{outputbox}
		\vspace{1cm}
		\begin{figure}
			\centering
			%\includegraphics[]{...}
			\caption{Testabbildung mit interner Referenz}
			\label{fig:abb1}
		\end{figure}
		\vspace{-1cm}
		Abbildung \ref{fig:abb1} zeigt nichts.	
	\end{outputbox}
\note{
- Label muss NACH der Caption stehen sonste funktioniert es nicht (oder nur fehlerhaft).+\\
- Analoge Funktionsweise für Tabellen}
\end{frame}
%-----------------------------------------------------------------------------------%
%---------------------------------------FRAME---------------------------------------%
%-----------------------------------------------------------------------------------%
\begin{frame}[fragile]
	\begin{Aufgabe}
		Füge ein Label zu der Tabelle \qquote{Vergleich der Körpereigenschaften} hinzu.
		
		Schreibe dann den folgenden Satz unter Verwendung von \lstinline[basicstyle=\normalfont\ttfamily\normalsize]|\ref{}|:
		
		\textrm{\qquote{Wenn nicht anders angegeben sind die Daten in Tabelle 1 von der Wissensmaschine WolframAlpha.}}
	\end{Aufgabe}

	\btVFill\Befehle
	\begin{center}
		\begin{tabular}{lp{6cm}}
			\toprule
			\LaTeX\ Befehl					&	Funktion					\\ \midrule
			\lstinline|\label{NAME}|		&	Weist einem Bestimmen Objekt (z.B. Abbildung, Tabelle, Liste oder Gleichung) einen \emph{internen} Name zu		\\
			\lstinline|\ref{}|				&	Referenziert das gegebene Objekt								\\
			\lstinline|\pageref{}|			&	Referenziert die Seite des Objekts								\\
			\lstinline|\eqref{}|			&	Setzt die Referenzierung in Klammern und schreibt sie aufrecht	\\
			\bottomrule
		\end{tabular}
	\end{center}
	\vspace{0.1cm}
\end{frame}
%-----------------------------------------------------------------------------------%
%---------------------------------------FRAME---------------------------------------%
%-----------------------------------------------------------------------------------%
\begin{frame}[fragile,shrink]
	\Losung
	\begin{outputbox}
		Wenn nicht anders angegeben sind die Daten in Tabelle \ref{tab:loewemensch} von der Wissensmaschine WolframAlpha.
		\vspace{-0.5cm}
		\begin{table}
			\centering
			\caption{Vergleich von Löwe und Mensch.}
			\label{tab:loewemensch}
			\vspace{-0.2cm}
			\begin{tabular}{r|cc}
				\hline
				\textbf{Eigenschaft}														&	\textbf{Löwe}	& \textbf{Mensch} \\\hline
				... &&
			\end{tabular}
		\end{table}
	\end{outputbox}
	
	\Code
	\begin{lstlisting}
Wenn nicht anders angegeben sind die Daten in Tabelle \ref{tab:loewemensch} von der Wissensmaschine WolframAlpha.	
\begin{table}
	\centering
	\caption{Vergleich von Löwe und Mensch.}
	\label{tab:loewemensch}
	\begin{tabular}{r|cc}
		...
	\end{tabular}
\end{table}
	\end{lstlisting}
\end{frame}

\end{document}