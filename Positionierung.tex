\documentclass["WS\space 16-17\space -\space LaTeX-Kurs\space -\space Praesentation\space -\space 2.tex"]{subfiles}

\begin{document}

\section{Positionierung von Abbildungen und Tabellen}
\begin{frame}[c]
	\begin{center}
		\LARGE \textbf{Positionierung von Abbildungen und Tabellen}
	\end{center}
\end{frame}
%%-----------------------------------------------------------------------------------------------%
%%------------------------------------------SUBSECTION-------------------------------------------%
%%-----------------------------------------------------------------------------------------------%
\subsection*{Positionierung}
%\begin{frame}
%	\begin{center}
%		\large Grundlagen
%	\end{center}
%\end{frame}
%-----------------------------------------------------------------------------------%
%---------------------------------------FRAME---------------------------------------%
%-----------------------------------------------------------------------------------%
\begin{frame}[fragile]
	
\vspace{0.1cm}
\begin{columns}
	\column{0.5\linewidth}
	\begin{lstlisting}
\begin{figure}[(!)h/b/t/p]
	\centering
	\includegraphics[]{}
	...
\end{figure}
\end{lstlisting}
	\column{0.5\linewidth}
	\begin{lstlisting}
\begin{figure}[H]
	\centering
	\includegraphics[]{}
	...
\end{figure}
	\end{lstlisting}
\end{columns}	
\vspace{-0.1cm}
%	\btVFill\Befehle
	\begin{center}
		\begin{tabular}{rl|  p{6.5cm} }
			\toprule
			Argument		&&	Positionierung																			\\ \midrule
		\rule{0pt}{9pt}	 \texttt{[h]}	& \textit{here} &	ungefähr (!) an der Stelle im Code 			\\
		\rule{0pt}{9pt}	 \texttt{[t]}	& \textit{top} &	am Anfang der Seite 																\\
		\rule{0pt}{9pt}	 \texttt{[b]}	& \textit{bottom}&	am Ende der Seite 																\\
		\rule{0pt}{9pt}	 \texttt{[p]}	& \textit{page}&	auf einer Extraseite für Abbildungen / Tabellen 									\\
		\rule{0pt}{9pt}	 \texttt{[!]}	&&	überschreibt interne Regeln für die Positionierung								\\
		\rule{0pt}{9pt}	 \texttt{[H]}	 & \textit{HERE} &	Platzierung an \emph{exakt} der Stelle im Code \\
			&&(benötigt das \texttt{float}-Package)	\\
			\bottomrule
		\end{tabular}
	\end{center}
	\vspace{0.1cm}
\note{
- mehrere Argumente möglich\\
- Reihenfolge in der LaTeX prüft: !, h, t, b (p ist extra)}
\end{frame}

\end{document}