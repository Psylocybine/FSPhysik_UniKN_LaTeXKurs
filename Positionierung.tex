\documentclass["WS\space 16-17\space -\space LaTeX-Kurs\space -\space Praesentation\space -\space 2.tex"]{subfiles}

\begin{document}

\section{Positionierung von Abbildungen und Tabellen}
\begin{frame}[c]
	\begin{center}
		\LARGE \textbf{Positionierung von Abbildungen und Tabellen}
	\end{center}
\end{frame}
%%-----------------------------------------------------------------------------------------------%
%%------------------------------------------SUBSECTION-------------------------------------------%
%%-----------------------------------------------------------------------------------------------%
\subsection*{Positionierung}
%\begin{frame}
%	\begin{center}
%		\large Grundlagen
%	\end{center}
%\end{frame}
%-----------------------------------------------------------------------------------%
%---------------------------------------FRAME---------------------------------------%
%-----------------------------------------------------------------------------------%
\begin{frame}[fragile]
	\begin{lstlisting}
\begin{figure}[(!)h/b/t/p/H]
	\centering
	\includegraphics[]{}
	...
\end{figure}
	\end{lstlisting}
	
	\btVFill\Befehle
	\begin{center}
		\begin{tabular}{l p{6cm} }
			\toprule
			Argument		&	Positionierung																			\\ \midrule
			\lstinline|[h]|	&	ungefähr an der Stelle im Code (allerdings nicht unbedingt genau dort) (here)			\\
			\lstinline|[t]|	&	am Anfang der Seite (top)																\\
			\lstinline|[b]|	&	am Ende der Seite (bottom)																\\
			\lstinline|[p]|	&	auf einer Extraseite für Abbildungen / Tabellen (page)									\\
			\lstinline|[!]|	&	überschreibt \LaTeX-interne Regeln für die Positionierung								\\
			\lstinline|[H]|	&	Platzierung an \emph{exakt} der Stelle im Code (benötigt das \texttt{float}-Package)	\\
			\bottomrule
		\end{tabular}
	\end{center}
	\vspace{0.1cm}
\note{
- mehrere Argumente möglich\\
- Reihenfolge in der LaTeX prüft: !, h, t, b (p ist extra)}
\end{frame}

\end{document}