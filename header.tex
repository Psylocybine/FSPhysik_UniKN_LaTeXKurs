\documentclass[xcolor=table,dvipsnames,8pt,t]{beamer}

\newcommand\hmmax{0}		%	damit ein son dummer Fehler nicht kommt :D
\newcommand\bmmax{0}

%--------------------------!-------------------------------------%
%--------------------------Notizen im PDF------------------------%
%----------------------------------------------------------------%

\setbeamertemplate{note page}[plain]										% Notizenseiten nur mit gesetztem Text
\AtBeginNote{\let\enumerate\itemize\let\endenumerate\enditemize\LARGE\par}	% Itemize in Notizen mit "Bullet", und \LARGE
\AtEndNote{\par}

%\setbeameroption{hide notes} % Only slides
%\setbeameroption{show only notes} % Only notes

%\usepackage{pgfpages}
%\setbeameroption{show notes on second screen=bottom} % Both

\makeatletter																% Zu jeder Slide eine Notizenseite
\def\beamer@framenotesbegin{% at beginning of slide
	\gdef\beamer@noteitems{}%
	\gdef\beamer@notes{{}}% used to be totally empty.
}
\makeatother

%--------------------------!-------------------------------------%
%--------------------------Schriftart----------------------------%
%----------------------------------------------------------------%

%\usepackage{mathptmx}
\usepackage{helvet}

%--------------------------!-------------------------------------%
%--------------------------Beamer Einstellungen------------------%
%----------------------------------------------------------------%

\usetheme{Dresden}
\usecolortheme{beaver}
\useinnertheme{circles} % not important. Comment if it breaks anything! Try

\setbeamertemplate{section in toc}	[sections numbered]
\setbeamertemplate{theorems}		[numbered]

\setbeamertemplate{itemize items}	[-]

\setbeamertemplate{caption}			[numbered]

%--------------------------!-------------------------------------%
%--------------------------Encoding------------------------------%
%----------------------------------------------------------------%

\usepackage[ngerman]{babel}			%	Einstellen der Sprache
\usepackage[T1]		{fontenc}		%	Wie wird Text ausgegeben, d.h. im PDF
\usepackage[utf8]	{inputenc}		%	Welche Zeichen 'versteht' LaTeX bei der Eingabe?
									%	Die beiden Pakete fontenc und inputenc gehören zwar nicht zusammen aber es wird empfohlen erst fontenc und dann inputenc zu laden
									%	http://tex.stackexchange.com/questions/44694/fontenc-vs-inputenc

%--------------------------!-------------------------------------%
%--------------------------Mathe---------------------------------%
%----------------------------------------------------------------%

\usepackage[fleqn]{mathtools}		%	Erweiterung von AMSMath, lädt automatisch AMSMath - für viele Mathe-Werkzeuge, fleqn ist für Mathe linksbündig
\usepackage{amssymb}				%	Für eine Vielzahl an mathematischen Symbolen

%--------------------------!-------------------------------------%
%--------------------------Listings------------------------------%
%----------------------------------------------------------------%

\usepackage{listings}
	\lstset{
		language		= [LaTeX]TeX,									%	[dialect]language
		extendedchars	= true,
		inputencoding	= utf8,
		upquote			= false,										%	benötigt textcomp bei true
		%
		backgroundcolor	= \color{blue!10},
		%
		basicstyle		= \ttfamily\footnotesize,
		identifierstyle	= ,
		commentstyle	= \color{gray},
		stringstyle		= \color{black},
		keywordstyle	= \color{green!70!black},						%	[] optional argument for level it applies to
		%directivestyle	= \normalfont\color{green},
		%
		showstringspaces= false,
		showspaces		= false,
		showtabs		= false,
		tab				= \rightarrowfill,
		tabsize			= 4,
		%
		linewidth			= \linewidth,
		breaklines			= true,
		breakautoindent		= true,
		breakindent			= 0pt,
		postbreak			= ,
		breakatwhitespace	= true,
		%
		%makemacrouse	= false,
		%
		aboveskip		= 0.2cm,
		belowskip		= 0.2cm,
		xleftmargin		= 0.1cm,
		xrightmargin	= 0.1cm,
		%
		frame			= single,									%	 none, leftline, topline, bottomline, lines (top and bottom), single, shadowbox ODER t,r,b,l
		%frameround		= ,
		framesep		= 0.20cm,
		rulesep			= 0.02cm,
		framerule		= 0.0cm,
%		framexleftmargin	= ,
%		framexrightmargin	= ,
%		framextopmargin		= ,
%		framexbottommargin	= ,
%		rulecolor		= ,
%		fillcolor		= ,
%		rulesepcolor	= ,	
		%
		columns			= fixed,
		fontadjust		= false,
		%
		%morestring		= [d][]$,										%	d - double, b - backslash
		morecomment		= [s][\color{orange}]{$}{$},					%	l - line, s - open and close,n - nested open and close, i - invisible
%		
		literate		= {Ö}{{\"O}}1
						  {Ä}{{\"A}}1
						  {Ü}{{\"U}}1
						  {ß}{{\ss}}1
						  {ü}{{\"u}}1
						  {ä}{{\"a}}1
						  {ö}{{\"o}}1
						  {\&}{{{\color{blue}{\&}}}}{1},
%
		morekeywords	= {part,chapter,section,subsection,subsubsection,paragraph},
		morekeywords	= {tableofcontents,listoffigures,listoftables},
		morekeywords	= {includegraphics,textSC,text,mathbb,eqref},,
		morekeywords	= {SI,SIrange,SIlist,num,per,tothe,metre,second,kilogram},
		morekeywords	= {printbibliography,citeauthor,citeyear,citetitle,psq,psqq},
		morekeywords	= {align},
		moredelim		= **[is][\only<+>{\color{red}}]{@}{@}
	}

%--------------------------!-------------------------------------%
%--------------------------SIUNITX-------------------------------%
%----------------------------------------------------------------%

\usepackage{siunitx}
	\sisetup{detect-all					= false}		%	Passt die Schriftart der gesetzen Einheiten / Zahlen an
	
	%\sisetup{detect-display-math		= true}
	%\sisetup{detect-family				= true}
	%\sisetup{detect-shape				= true}
	%\sisetup{detect-weight				= true}
	
	\sisetup{group-separator			=	\,}			%	Zeichen Zwischen Dreiergruppen in langen Zahlen (1.000.000,5)
	
	\sisetup{separate-uncertainty		=	false}		%	Unsicherheiten mit +- (true) oder mit anderen Zeichen (false)
	\sisetup{uncertainty-separator		=	\,}			%	Zeichen zwischen Zahl und Unsicherheit bei seperate uncertainty=false
	\sisetup{output-open-uncertainty	=	(}			%	Zeichen vor Unsicherheit bei seperate uncertainty=false
	\sisetup{output-close-uncertainty	=	)}			%	Zeichen nach Unsicherheit bei seperate uncertainty=false
	
	\sisetup{list-final-separator	=	{ und }}		%	Finales Trennzeichen bei Listen
	\sisetup{list-pair-separator	=	{ und }}		%	Trennzeichen bei Listen bestehend aus zwei Teilen
	\sisetup{list-separator			=	{, }}			%	Trennzeichen bei Listen aus 3 oder mehr Teilen
	\sisetup{range-phrase			=	{ bis }}		%	Text zur angabe von Bereichen (500nm bis 700nm)
	
	\sisetup{inter-unit-product	=	\ensuremath{\,}}	%	Zeichen zwischen Einheitenzeichen
	
	\sisetup{per-mode					=	reciprocal-positive-first}	%	Ausgabe von Einheiten (symbol-or-fraction, fraction, symbol, reciprocal-positive-first, repeated-symbol)
	\sisetup{per-symbol					=	/}			%	Zeichen zwischen Zähler und Nenner bei Option \symbol
	\sisetup{bracket-unit-denominator	=	false}		%	Klammern um Zähler und Nenner oder nicht
	
	\sisetup{number-unit-product=	\ensuremath{\,}}	%	Symbol zwischen Zahl und Einheit
	
	\sisetup{multi-part-units			=	brackets}	%	Angabe der Einheit bei Zahlen mit Unsicherheit (z.B. mit '+') (brackets, repeat, single)
	\sisetup{product-units				=	repeat}		%	Angabe der Einheit bei Produkten (Eingabe mit 'x' oder '/') (brackets, repeat, single)
	\sisetup{list-units					=	repeat}		%	Angabe der Einheit bei Listen
	\sisetup{range-units				=	repeat}		%	Angabe der Einheit bei Bereichen 
		
	\sisetup{exponent-product			=	\cdot}		%	Zeichen bei 1*10^5 zwischen Zahl und 10^5
	
	%\sisetup{locale = DE}								%	Automatische Einstellung der Ausgabe für bestimmte Regionen (UK, US, DE, FR, ZA)
	%\sisetup{strict = true}							%	Überschreibt alle manuell gesetzten Einstellungen und benutzt nur die durch 'locale' gesetzen Werte

%--------------------------!-------------------------------------%
%--------------------------Abstände------------------------------%
%----------------------------------------------------------------%

\setlength{\parindent}				{0cm}							%	Längenangabe für die Einrückung der ersten Zeile eines neuen Absatzes.
\setlength{\parskip}				{2mm plus 1mm minus 1mm}		%	Längenangabe für den Abstand zwischen zwei Absätzen.

%--------------------------!-------------------------------------%
%--------------------------Color---------------------------------%
%----------------------------------------------------------------%

\usepackage{color}

%--------------------------!-------------------------------------%
%--------------------------TikZ----------------------------------%
%----------------------------------------------------------------%

\usepackage{tikz} % needed for opaqueblock-definition needed for hiding colorboxes

% \usepackage[table]{xcolor} %needed for command cellcolor %moved into beamer-option

\tikzset{visib/.style={rectangle,color=blue,fill=blue!10,text=black,text opacity=1, text width=#1,align=flush center}}
\tikzset{invisib/.style={rectangle,color=white,fill=white,text=black,text opacity=0, text width=#1,align=flush center}}

\newenvironment{fancycolorbox}{\begin{center}
\begin{tikzpicture}}{\end{tikzpicture}
\end{center}}

\newcommand{\opaqueblock}[4]{
\node<#1>[#2=#3] (X) {#4};
}

%--------------------------!-------------------------------------%
%--------------------------Chemisches Zeug-----------------------%
%----------------------------------------------------------------%

\usepackage{chemmacros}
\usechemmodule{all}

%--------------------------!-------------------------------------%
%--------------------------Akronyme------------------------------%
%----------------------------------------------------------------%

\usepackage{acro} 

\DeclareAcronym{D}{
      short = D ,
      long = Wüste ,
      long-plural = n ,
      foreign = Desert ,
      short-format = \scshape 
      } 

%--------------------------!-------------------------------------%
%--------------------------Picture Stuff-------------------------%
%----------------------------------------------------------------%

%for watermarks (distinguish between math and text)
  
%\usepackage[left=1in,right=1in]{geometry}
%\usepackage{xcolor,rotating,picture}

%--------------------------!-------------------------------------%
%--------------------------Tabellen------------------------------%
%----------------------------------------------------------------%

\usepackage{booktabs}

%--------------------------!-------------------------------------%
%--------------------------Literatur-----------------------------%
%----------------------------------------------------------------%

\usepackage{csquotes}									%	Wird von BibLaTeX benötigt
\usepackage[backend=biber,style=numeric]{biblatex}		%	Für die Literaturverwaltung
	\addbibresource{Literaturverzeichnis.bib}			%	Laden eures Literaturverzeichnisses

% enable compiling tex-files that are "inputed" via \subfile-macro and subfiles-documentclass
\usepackage[]{subfiles}

% only used for \BibTex command
\usepackage{hologo}
  
\setbeamertemplate{bibliography item}[text]

%--------------------------!-------------------------------------%
%--------------------------Newcommands---------------------------%
%----------------------------------------------------------------%

%\renewcommand{\epsilon}{\varepsilon}
\newcommand{\textSC}[1]{{\normalfont  \textsc{#1}}} 
\renewcommand{\quote}[1]{\glq{#1}\grq}        			%	Um Worte EINFACH in Anführungszeichen zu setzen
\newcommand{\qquote}[1]{\glqq{#1}\grqq}  				%	Um Worte EINFACH in doppelte Anführungszeichen zu setzen

\newcounter{AufgCounter}
\setcounter{AufgCounter}{1}
\numberwithin{AufgCounter}{section}
\resetcounteronoverlays{AufgCounter}

\definecolor{AufgColor}{HTML}{52B0E3}

\newcommand{\Aufgabee}	{\refstepcounter{AufgCounter}	\textbf{{\large 	 \color{AufgColor} Aufgabe 		\theAufgCounter:}}\vspace{0.1cm}\newline}
\newcommand{\Losung}	{								\textbf{{\normalsize \color{AufgColor} L{\"o}sung 	\theAufgCounter:}}\vspace{-0.3cm}\newline}
\newcommand{\Code}		{								\textbf{{\normalsize \color{AufgColor} Code 		\theAufgCounter:}}\vspace{-0.4cm}\newline}

\newcommand{\Ausgabe}	{\textbf{Ausgabe:}\vspace{0.1cm}\newline}
\newcommand{\Befehle}	{\textbf{Befehle:}\vspace{-0.3cm}}

\newcommand{\btVFill}{\vskip0pt plus 1filll}			%	Um Sachen ans Ende der Folie zu schreiben

\newcommand{\linebreakrule}{\vspace{-0.35cm}\makebox[\textwidth]{\rule{\textwidth}{0.2mm}}}

%\newcolumntype{K}[1]{>{\centering\arraybackslash}p{#1}}

%--------------------------!-------------------------------------%
%--------------------------Newenvironment------------------------%
%----------------------------------------------------------------%

\newtheorem{thmex}	{Aufgabe}	[section]

\newenvironment{Aufgabe}
	{\begin{thmex}\vspace{-0.1cm}}
	{\end{thmex}}

\setbeamercolor{block body example}	{bg=yellow!10}
\setbeamercolor{postit}				{fg=black,bg=yellow!10}

\newenvironment{outputbox}
	{\vspace{0.3cm}\ \\\begin{beamercolorbox}[sep=0cm,colsep=0.1cm,colsep*=0.1cm,wd=\textwidth]{postit}\vspace{-0.3cm}\\\normalfont}
	{\end{beamercolorbox}}